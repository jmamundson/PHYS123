\documentclass[11pt,letterpaper]{article}
\usepackage{array}
\usepackage{fullpage}
\usepackage{verbatim}
\usepackage{parskip}
\usepackage{graphicx}
\usepackage{url}
	\usepackage{xcolor,colortbl}
	\definecolor{gray}{rgb}{0.95,0.95,0.95}
	\newcommand{\gray}{\cellcolor{gray}}  %{0.9}
\usepackage[flushmargin]{footmisc}

%%%%%%%%%%%%%%%%%%%%%%%%%%%%%%%%%%%%%%%%%%%%
\usepackage{termcal}
% Few useful commands (our classes always meet either on Monday and Wednesday 
% or on Tuesday and Thursday)

\newcommand{\MTWFClass}{%
\calday[Monday]{\classday} % Monday
\calday[Tuesday]{\classday} % Tuesday
\calday[Wednesday]{\classday} % Wednesday
\skipday % Thursday (no class)
\calday[Friday]{\classday} % Friday 
\skipday\skipday % weekend (no class)
}

\newcommand{\holiday}[2]{%
\options{#1}{\noclassday}
\caltext{#1}{#2}
}

\newcommand{\lab}[2]{%
\options{#1}{\noclassday}
\caltext{#1}{#2}
}


\renewcommand{\calprintdate}{%
     \ifnewmonth\framebox{\arabic{month}/\arabic{date}}%
     \else\arabic{month}/\arabic{date}%
     \fi}
         
%%%%%%%%%%%%%%%%%%%%%%%%%%%%%%%%%%%%%%%%%%%%
\renewcommand{\thefootnote}{\fnsymbol{footnote}}

\newcommand{\squeezeup}{\vspace{-2.5mm}}
\setlength{\parindent}{0in}
\newcommand{\tablespace}[0]{\vspace{8pt}}

\begin{document}
\begin{centering}
\textbf{PHYS S123: College Physics I}

Fall 2025\\
\hfill{}\\

\bigskip
\begin{table}[h]
\centering
\setlength{\extrarowheight}{2pt}
\squeezeup
\begin{tabular}{@{}r@{\hspace{0.1in}}p{4.25in}} 
{\bf Instructor:} & Jason Amundson\\
& {\'A}ak'w T{\'a} H{\'i}t 209 \\
& jmamundson@alaska.edu\\
& phone: 796-6247 \tablespace\\
{\bf Class hours:} & MWF 10:45 am -- 11:45 am \tablespace\\
{\bf Lab hours:} & J01, T 8:45 am -- 11:45 am\\
& J02, T 1:15 pm -- 4:15 pm \tablespace\\
{\bf Office hours:} & MWF 12:00 pm -- 1:00 pm, or by appointment\tablespace\\
{\bf Website:} & A course website will be maintained on Blackboard (http://classes.alaska.edu). Check for assignments, handouts, grades, and messages.\tablespace\\
{\bf Prerequisites:} & MATH S152\tablespace\\
{\bf Textbook:} & College Physics: A Strategic Approach (4\textsuperscript{th} ed.) by Knight, Jones, and Field. Be sure to purchase a version of the book that comes with an access card for MasteringPhysics, which you will use for homework submissions.
\tablespace\\
& The cheapest option is to purchase the MasteringPhysics with Pearson eText package (ISBN-13: 978-0-13-470393-0 for 24-month access; ISBN-13: 978-0-13-678221-6 for single-term access). If you prefer, you can also order a package that includes a bound copy of the textbook. 
\tablespace\\
& To access MasteringPhysics, follow the link from the course Blackboard site and click ``Open MyLab and Mastering''. From there you will need to enter the access code that you received when you purchased the textbook.  \tablespace\\
{\bf Other materials:} & A basic scientific calculator with trigonometric, exponential, and logarithmic functions. Calculators can be used during exams.
\end{tabular}
\end{table}
\end{centering}

\textbf{Student Learning Outcomes}\\
Upon successful completion of this course, students will be able to:
\begin{enumerate}\itemsep -5pt
\item Demonstrate an understanding of the basic laws of physics in classical mechanics and thermodynamics.
\item Apply these physics laws to understand physical phenomena and technological applications.
\item Demonstrate quantitative physics problem solving skills through the application of algebra and critical physics thinking.
\item Describe the societal relevance of physics and its connection to other fields of science.
\item Safely use basic laboratory equipment, develop a testable hypothesis, systematically collect and analyze data, and report and interpret experimental results.
\end{enumerate}\bigskip

Physics is the study of matter and its motion through space and time. In PHYS S123 we will cover the field of classical mechanics, which pertains to ``large'' objects that are moving much slower than the speed of light.


\begin{figure}[h]
\begin{center}
\includegraphics[width=6.5in, trim={0 2in 0 0}, clip]{./flowchart.pdf}
\end{center}
\end{figure}

\clearpage
\textbf{Grading} 
\begin{table}[h!]
\squeezeup
\begin{tabular}{ll}
Homework assignments & 20\%\\
Informal lab reports (8) & 20\%\\
Formal lab reports (2) & 15\%\\
Exams (3) & 45\%
\end{tabular}
\end{table}

\textbf{Homework}\\
There will be 10 homework assignments consisting of roughly 10 problems each. You will submit your solutions through MasteringPhysics. Late assignments will only be accepted for extenuating circumstances.

Why use MasteringPhysics?
\begin{itemize}\itemsep -5pt
\item It gives you wrong-answer feedback and hints for solving problems.
\item It provides me with diagnostics on what types of problems are giving you the most trouble.
\end{itemize}

A few suggestions for working through the homework:
\begin{itemize}\itemsep -5pt
\item Work through your answers slowly and be sure to check for significant figures before entering your solution. You might consider solving all of the problems before entering your answers so that you don't get frustrated right away.
\item If you feel confident in your answer, but MasteringPhysics tells you that your answer is incorrect and doesn't give you useful feedback, send me an e-mail. View this as an opportunity to figure out what you did wrong and correct your mistake before the due date.%, which is an option that you wouldn't have with paper submission.
\item I view the homework as training for the exams, and in that sense the homework grade is essentially participation. I'm happy to help you work through problems all the way to the final solution if you are having trouble. 
\item I reserve time in class each week to address questions related to the homework, so come with questions!
\end{itemize}

\textbf{Lab reports}\\
There will also be 10 laboratory exercises during the semester. You will be expected to turn in informal reports for eight of the labs and formal reports for the other two. For the informal reports you may submit a group report (no more than three people in a group) but are not required to do so. For the formal reports I expect you to submit an individual report. More details to follow. You may submit physical or electronic copies of your reports; if you choose to submit a report electronically you should do so by uploading it to Blackboard.

Lab reports will not be considered late up until the point that I grade them; afterwards they will be docked by 50\%. This late policy is designed to (1) encourage you to finish your reports even if they were not completed before the due date and (2) simplify my grading. I don't like taking off points for late work, especially if I am slow at grading it, but it is also much easier to grade everybody's work at once. I think this policy is a good compromise. 
\\

\textbf{Exams}\\
You will be given three exams during the semester, including the final exam. The exams will focus on the material that was covered over the previous third of the semester, but they are cumulative in the sense that everything that we do in physics will be built up from the same core principles. The exams will be proctored in the Learning Center. You will have several days to complete them. The exam scores will be curved.\\

%\clearpage
\textbf{Grading Scale}
\begin{table}[h!]
\squeezeup
\begin{tabular}{ll}
A & 93--100\% \\
A- & 90--92\% \\
B+ & 87--89\% \\
B & 83--86\% \\
B- & 80--82\% \\
C+ & 77--79\% \\
C & 73--76\% \\
C- & 70--72\% \\
D+ & 67--69\% \\
D & 63--66\% \\
D- & 60--62\% \\
F & $<$60\%
\end{tabular}
\end{table}

%I may lower this grading scale if I decide that the course assignments have been too difficult. I will not do the opposite. \\

%\textbf{General Comments}\\
%My job is to help you learn. If you are uncomfortable in the classroom or have any other comments and concerns, please do not hesitate to contact me.

%When I'm in the office I'll try to respond to your phone calls and e-mails as quickly as possible. However, in general I do not respond to messages after 6:00 pm or on weekends. Any messages sent to me during that time might not be addressed until the following morning.\\

%\textbf{Conduct}\\
%Please turn off cell phones, laptops, and other electronic devices (except calculators) during lecture unless you need them to aid. They can be a distraction to you and your fellow students.\\

\textbf{Student Ratings of Instruction}\\
During the last three weeks of class, you will have an opportunity to complete an online rating questionnaire on course instruction, how the course aided in your skill development,  effectiveness of technology and equipment used, and adequacy of library resources and services used during the course. You will receive notification in your UAS email account when the rating questionnaire is available. Please make use of this opportunity to provide feedback on what worked for you and what did not. Your input is used to assess methods and services in order to provide the best educational experience possible.\\

\textbf{Disabilities}\\
If you experience a disability and would like information about support services, please contact Disability Services, located at the Student Resource Center in the Mourant building.  They can be reached at 796-6000. For more information, please see http://www.uas.alaska.edu/dss/index.html.\\

\textbf{Title IX/Sexual Misconduct}\\
All students have the right to be free from all forms of gender and sex-based misconduct (sexual harassment, dating violence, domestic violence, sexual assault, or stalking). Please report any incidence of sex or gender-based discrimination to the UAS Title IX Office: \url{https://uas.alaska.edu/equity-and-compliance/titleix/index.html}

\clearpage
\paragraph*{Schedule (subject to change):}
\begin{center}
\begin{calendar}{8/25/2025}{16} 
% Semester starts on 8/23/2021 and lasts for 16
% weeks, including finals week

\setlength{\calboxdepth}{.6in}
\renewcommand{\calprintclass}{}
\MTWFClass

% August
\caltexton{1}{Course overview: why physics?}
\caltextnext{1.1--1.6: Introduction to kinematics}
\caltextnext{2.1--2.6: Kinematic equations}


% September
%\caltextnext{\gray TBD}
\caltextnext{\gray 3.1--3.3, 3.5: Motion in two dimensions\\{\bf HW \#1 due}}

\caltextnext{\gray 3.4--3.6: Motion in two dimensions, ctd.}
\caltextnext{\gray 3.8: Relative motion}
\caltextnext{\gray 3.7, 6.1: Circular motion\\{\bf HW \#2 due}}

\caltextnext{\gray 7.1--7.2: Rotational motion}
\caltextnext{\gray Circular and rotational motion, ctd.}
\caltextnext{\gray 4.1--4.7: Newton's Laws}

\caltextnext{\gray 5.3--5.4, 5.8, 6.6: Types of forces, I}
\caltextnext{\gray 5.5: Types of forces, II\\{\bf HW \#3 due}}
\caltextnext{\gray {\bf Exam \#1}\\Testing Center}

\caltextnext{\gray 5.6, 8.3: Types of forces, III}
\caltextnext{6.3--6.4, 6.6: Centripetal forces}


\caltextnext{7.3--7.4: Torque}
\caltextnext{7.5--7.7: Torque, II}
% October
\caltextnext{8.1--8.2 Static equilibrium\\{\bf HW \#4 due}}

\caltextnext{8.4: Equilibrium and elasticity}
\caltextnext{9.1--9.3: Impulse and momentum}
\caltextnext{9.4--9.6: Conservation of momentum\\{\bf HW\#5 due}}

\caltextnext{9.7: Angular momentum}
\caltextnext{10.1--10.2: Work and energy}
\caltextnext{10.3--10.5: Types of energy\\{\bf HW\#6 due}}

\caltextnext{10.6--10.7: Types of energy, II}
\caltextnext{11.1, 11.4--11.7, 12.5: Introduction to thermodynamics}
% November
\caltextnext{12.6: Applications of thermodynamics\\{\bf HW \#7 due}}

\caltextnext{{\bf Exam \#2}\\Testing Center}

\caltextnext{\gray 12.8: Heat transfer}
\caltextnext{\gray 12.2--12.4, 12.7: Thermodynamics of gases}

\caltextnext{\gray 13.1--13.2: Introduction to fluids}
\caltextnext{\gray 13.3: Archimedes' principle}

\caltextnext{\gray 13.4--13.5: Fluid dynamics, I\\{\bf HW \#8 due}}

\caltextnext{\gray 13.6--13.7: Fluid dynamics, II}
\caltextnext{\gray 14.1--14.5: Oscillations}

\caltextnext{\gray 14.6--14.7: Driven and damped oscillations\\{\bf HW \#9 due}}
\caltextnext{\gray 15.1--15.3: Introduction to waves}
\caltextnext{\gray 15.4--15.7: Traveling waves}

% December
\caltextnext{16.1--16.3: Wave superposition}
\caltextnext{16.4--16.5: Sound\\{\bf HW \#10 due}}
\caltextnext{TBD}


% Field work
\holiday{9/22/2025}{\gray No class}


% Labs
\lab{8/26/2025}{No lab}
\lab{9/2/2025}{\gray Lab \#1: Measurements and motion}
\lab{9/9/2025}{\gray Lab \#2: Acceleration\\{\bf Lab \#1 due}}
\lab{9/16/2025}{\gray Lab \#3: Forces\\{\bf Lab \#2 due}}
\lab{9/23/2025}{\gray Review for exam \#1\\{\bf Lab \#3 due}}

\lab{9/30/2025}{\gray Lab \#4: Circular motion (formal report)}

\lab{10/7/2025}{Lab \#5: Torque \\{\bf Lab \#4 due}}
\lab{10/14/2025}{Lab \#6: Statics \\{\bf Lab \#5 due}}
\lab{10/21/2025}{Lab \#7: Momentum and energy\\{\bf Lab \#6 due}}
\lab{10/28/2025}{Review for exam \#2}

\lab{11/4/2025}{\gray Lab \#8: Springs (formal report)\\{\bf Lab \#7 due}}
\lab{11/11/2025}{\gray No lab}
\lab{11/18/2025}{\gray Lab \#9: Pendulums\\{\bf Lab \#8 due}}
\lab{11/25/2025}{\gray Lab \#10: Sound\\{\bf Lab \#9 due}}
\lab{12/2/2025}{Review for exam \#3\\{\bf Lab \#10 due}}

% Holidays
\holiday{9/1/2025}{\gray Labor Day}
\holiday{11/26/2025}{\gray Fall break}
\holiday{11/28/2025}{\gray Fall break}
\holiday{12/8/2025}{Exam \#3 - - - - - - - -  \\ Testing center}
\holiday{12/9/2025}{- - - - - - - - - - - - - - -}
\holiday{12/10/2025}{- - - - - - - - - - - - - - -}
\holiday{12/12/2025}{- - - - - - - - - - - - - - $>$}



\end{calendar}
\end{center}


\end{document}
