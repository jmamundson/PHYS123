\documentclass[11pt,letterpaper]{article}
\usepackage{array}
\usepackage{fullpage}
\usepackage{graphicx}
\usepackage{parskip}
\usepackage{amsmath}
\usepackage[small]{caption}
\usepackage{graphpap}
\usepackage{logpap}
\usepackage{tabularx}
\usepackage{url}
\usepackage{hyperref}
\usepackage{enumitem}

\renewcommand{\thesection}{PART \arabic{section}: }

\newcounter{question}[section]
\newenvironment{question}[1][]{\refstepcounter{question}\par\medskip
   \textbf{\arabic{section}.\thequestion.} \rmfamily}{\medskip}

\usepackage{titlesec}
\titleformat{\section}{\clearpage\normalfont\bfseries}{\thesection}{0em}{}
\titlespacing{\section}{0pt}{0.5\baselineskip}{0pt}

\titleformat{\subsection}[runin]
{\normalfont\bfseries}{\thesubsection}{1em}{}

\titleformat{\subsubsection}{\normalfont\bfseries}{\thesubsubsection}{0em}{}
\titlespacing{\subsubsection}{0pt}{0.5\baselineskip}{0pt}

\newcounter{saveenumi}
\newcommand{\seti}{\setcounter{saveenumi}{\value{enumi}}}
\newcommand{\conti}{\setcounter{enumi}{\value{saveenumi}}}

\usepackage[dvipsnames]{xcolor}
\newcommand{\sol}[1]{{\color{NavyBlue} #1}}


\begin{document}
\setlength{\parindent}{0in}

\begin{flushright}
PHYS S123\\
Lab 2: Acceleration and gravity\\
9/9/25 (due 9/16/25)
\end{flushright}

Name(s):\\

\subsubsection*{Topics:}
\begin{enumerate}
\setlength{\parskip}{3pt}
\item Acceleration along a horizontal track
\item Determination of gravity using kinematic equations
\end{enumerate}

\subsubsection*{Introduction:}
This lab is designed to give you a qualitative understanding of acceleration, which is less intuitive than position and velocity. The lab will also teach you how to quantitatively describe motion involving acceleration. Finally, you will discover that gravitational acceleration is essentially constant near the Earth's surface; you will try to determine that constant using the kinematic equations. Recall that the two fundamental kinematic equations for constant acceleration are:
$$\Delta{x}=v_i\Delta{t}+\frac{1}{2}a\Delta{t}^2$$
$$v_f=v_i+a\Delta{t}$$
These two equations can be combined to form another useful equation:
$$v_f^2-v_i^2=2a\Delta{x}$$

You will use the Logger Pro Motion Sensor to collect your data for this lab, and MATLAB to generate plots.\\

\subsubsection*{What you should turn in:}
Include the following in a single document separate from the lab hand-out. You can turn in a group report. Be sure to include labels and units on all graphs. 
\begin{enumerate}
\setlength{\parskip}{3pt}
\item Section 1.2: One velocity-time and one acceleration-time graph with short descriptions [2 pts]
\item Section 1.3: One velocity-time and one acceleration-time graph with short descriptions [2 pts]
\item Section 1.4: Discussion of numerical derivatives [4 pts]
\item Section 2.1: One position-time and one velocity-time graph, and calculations of $g$ [4 pts]
\item Section 2.2: Measurements and calculations of $g$ [4 pts]
\item Section 2.3: A derivation showing how you can calculate your reaction time; results of your experiments [2 pts]
\item Section 3: Answers to questions [4 pts]
\end{enumerate}

\subsubsection*{Equipment:}
\begin{itemize}
\setlength{\parskip}{3pt}
\item LabQuest interface, cables, and motion sensor
\item Track and cart
\item Fans
\item AA batteries/blanks
\item Photogates
\item Black rubber ball
\end{itemize}

\textbf{Some tips for working with Logger Pro and MATLAB:}
\begin{itemize}
\item Data collected by Logger Pro can be exported as text files, which can then be imported into MATLAB. The text file will contain five header lines followed by two columns of data.

\item There are a few ways to import text files into MATLAB. One way is to click file $>$ importdata and then play with the options to turn the data columns into a MATLAB variable. A second, more powerful way is to use the \verb+importdata+ command. In order to use \verb+importdata+, you must first change the MATLAB directory to the directory containing the text file. Then type 
\begin{verbatim}>>A=importdata(`file.txt');\end{verbatim} 
where \verb+file.txt+ is the name of the text file. \verb+data+ will be matrix containing the columns of \verb+file.txt+. If the columns correspond to time, position, and velocity, you could set
\begin{verbatim}>>t=A.data(2:end,1);\end{verbatim}
This creates a variable \verb+t+ using all of the rows of \verb+data+ but just the first column. Similarly,
\begin{verbatim}>>x=A.data(2:end,2);\end{verbatim}
\begin{verbatim}>>v=A.data(2:end,3);\end{verbatim}

\item You can make plots in several ways. As you saw in the first lab, \verb+plot(t,x)+ will make a plot with \verb+t+ on the horizontal axis and \verb+x+ on the vertical axis. If you'd like to make a couple of plots in one figure, you could do
\begin{verbatim}>>subplot(2,1,1)\end{verbatim}
\begin{verbatim}>>plot(t,x)\end{verbatim}
\begin{verbatim}>>subplot(2,1,2)\end{verbatim}
\begin{verbatim}>>plot(t,v)\end{verbatim}
Another option is to use \verb+plotyy+, which puts two plots on the same graph with two vertical axes.
\begin{verbatim}>>plotyy(t,x,t,v)\end{verbatim}

You may wish to plot multiple data sets on the same axes. There are a few ways to do this. Let \verb+t+ be a time vector, \verb+v+ be a velocity vector that you calculated, and \verb+velocity+ be a velocity vector that you imported from LoggerPro. You can compare your calculation to LoggerPro's by doing either
\begin{verbatim}>> plot(t,v);\end{verbatim}
\begin{verbatim}>> hold all;\end{verbatim}
\begin{verbatim}>> plot(t,velocity};\end{verbatim}
\begin{verbatim}>> hold off;\end{verbatim}
or by typing
\begin{verbatim}>> plot(t,v,t,velocity);\end{verbatim}

Also, there are command line options for editing figures, but for now I recommend using the GUI (graphical user interface).

\end{itemize}

\section{ACCELERATION ALONG A HORIZONTAL TRACK}
Set up your track on the floor or on a table. Use an inclinometer (or a marble) to make sure that the track is level. Place a motion detector at one end of the track. Use a position graph to make sure that the detector can detect the cart at the end of the track.

\question{Preliminary investigation: Give the cart a push and watch its subsequent motion. After you let go does it speed up, slow down, or move uniformly? (No need to comment on this in your report.)}

\question{Now attach a motorized fan to the cart; the fan will cause the cart to accelerate without manual intervention. The fan accepts four batteries, but aluminum ``blanks'' may also be used to produce less acceleration. Try starting with just one or two batteries for more controlled experiments.

Turn the fan on and track the cart's motion with the motion detector. Try orienting the fan so that it is blowing both toward and away from the detector. \textbf{Submit velocity and acceleration plots for one direction of motion.} Plot the velocity and acceleration that are output from LoggerPro, but also calculate the velocity and acceleration yourself and plot the results on the same graphs. For example, if the time vector is \verb+t+ and the position data is \verb+x+, you can calculate the velocity and acceleration using the \verb+gradient+ function.
\begin{verbatim}>> dt = t(2)-t(1); % time step\end{verbatim}
\begin{verbatim}>> v = gradient(x,dt); % compute the velocity\end{verbatim}
\begin{verbatim}>> a = gradient(v,dt); % compute the acceleration\end{verbatim}

Write short descriptions for both sets of plots, being sure to address the following questions:
\begin{enumerate}
\item Is the cart's acceleration positive or negative? How does the direction of motion affect the sign of acceleration? (HINT: Remember that acceleration is the rate of change of velocity.)
\item How does the cart's velocity vary with time? Does it increase at a steady rate or in some other way?
\item How does the acceleration vary with time?  
\item How do the position-time and velocity-time graphs compare to that of a uniformly moving object?\\
\end{enumerate}}

\question{Now give the cart a shove toward the detector while the fan is blowing at the detector. The cart will slow down after it is released. After you release the cart, is the acceleration positive or negative? \textbf{Submit velocity and acceleration plots as in 1.2.}

Write a short description of this plot. In addition, state a general rule (either with words or a mathematical formula) relating the sign of the acceleration, the direction of motion, and knowledge of whether the cart is speeding up or slowing down.}
 
%\question{Search on MathWorks (\verb+http://www.mathworks.com/help/matlab/+) to figure out how MATLAB calculates derivatives using the \verb+diff+ and \verb+gradient+ commands. Briefly summarize the difference. How do the velocity vectors computed by LoggerPro compare to what you calculated? Does LoggerPro use something like \verb+diff+ or \verb+gradient+? Or something different?}

\section{GRAVITATION ACCELERATION}
In this section of the lab you will attempt to determine the acceleration due to gravity, $g$, experimentally.

\question{Tilt the cart track and, after measuring the tilt angle with an inclinometer (or using trigonometry), release the cart from rest. The cart will roll down the incline as it is pulled by gravity. The component of gravity that is parallel to the track (the cart's direction of motion) is $g\sin\theta$, which represents the acceleration of the cart (neglecting friction). \textbf{Remove any excess data collected before the cart starts moving or after it stops moving. Then plot the position-time and velocity-time graphs of this motion.} Use \verb+polyfit+ to fit a quadratic equation through your position data. To use \verb+polyfit+,
\begin{verbatim}>> coeffs = polyfit(t,x,2) % calculate coefficients of a 2nd order polynomial\end{verbatim}
Note that I suppressed the semicolon in order to display the coefficients. What do the coefficients represent? What is $g$? To help your interpret the output of \verb+polyfit+, you can try typing
\begin{verbatim}>> help polyfit\end{verbatim}}

\question{Mount two photogates to a stand and daisy chain (string) them together, one above the other. Connect them to the LabQuest interface. The photogates work by recording the time at which the ``gates'' are blocked and unblocked. They have both internal and external gates. Make sure that the internal gates are open. After pressing collect, drop one of the black rubber balls through the top gate. You should see four times appear: the times that the first gate is blocked and then unblocked again, and then the times that the second gate is blocked and then unblocked. Use this information to calculate the ball's free fall acceleration.}

\question{One of the characteristics of physics is that knowledge of one aspect of nature may permit measurements of completely unrelated phenomena.  For instance, knowledge of the acceleration of gravity can be used to measure the height of a building or the speed of a projectile.  You may find your ``reaction time'' with a falling meter stick.  Your partner holds one end of the meter stick and you position your fingers around the 50~cm mark.  Your partner releases the stick at some arbitrary time.  The distance it falls before you catch it can be used to find your reaction time.  (HINT: Use the equations of motion to derive a way of calculating time if you know the acceleration and distance.)  Repeat the measurement several times to get an average.

What is your average reaction time?  Is there a difference between the reaction time of your two hands?}


\section{DISCUSSION}
Answer the following questions in your lab report.

1a.  When an object has positive velocity and is slowing down, is the acceleration positive, zero, or negative?

1b.  When an object has negative velocity and is slowing down, is the acceleration positive, zero, or negative?

1c.  When an object has positive velocity and is speeding up, is the acceleration positive, zero, or negative?

1d.  When an object has negative velocity and is speeding up, is the acceleration positive, zero, or negative?\\

2. How can you tell from a velocity-time graph that a moving object has changed directions?\\

3. How did neglecting air resistance affect your measurements and calculations in part 2.2? How would you expect the results to change if you used a different shape or density for the falling object?

\end{document}
