\documentclass[11pt,letterpaper]{article}
\usepackage{array}
\usepackage{fullpage}
\usepackage{graphicx}
\usepackage{parskip}
\usepackage{amsmath}
\usepackage[small]{caption}
\usepackage{graphpap}
\usepackage{logpap}
\usepackage{tabularx}
\usepackage{url}
\usepackage{hyperref}
\usepackage{enumitem}

\renewcommand{\thesection}{PART \arabic{section}: }

\newcounter{question}[section]
\newenvironment{question}[1][]{\refstepcounter{question}\par\medskip
   \textbf{\arabic{section}.\thequestion.} \rmfamily}{\medskip}

\usepackage{titlesec}
\titleformat{\section}{\clearpage\normalfont\bfseries}{\thesection}{0em}{}
\titlespacing{\section}{0pt}{0.5\baselineskip}{0pt}

\titleformat{\subsection}[runin]
{\normalfont\bfseries}{\thesubsection}{1em}{}

\titleformat{\subsubsection}{\normalfont\bfseries}{\thesubsubsection}{0em}{}
\titlespacing{\subsubsection}{0pt}{0.5\baselineskip}{0pt}

\newcounter{saveenumi}
\newcommand{\seti}{\setcounter{saveenumi}{\value{enumi}}}
\newcommand{\conti}{\setcounter{enumi}{\value{saveenumi}}}

\usepackage[dvipsnames]{xcolor}
\newcommand{\sol}[1]{{\color{NavyBlue} #1}}



\begin{document}
\setlength{\parindent}{0in}

%% EQUIP: Force Probe, Pulley, Track, Hangar, Weight, Tilt, Scale

\begin{flushright}
PHYS S123: College Physics I\\
Lab 7: Momentum and Energy\\
10/21/25 (due 10/28/24)
\end{flushright}

Name(s):\\


\subsubsection*{Topics:}
\begin{enumerate}
\setlength{\parskip}{3pt}
\item Qualitative analysis of collisions
\item Inelastic collisions
\item Elastic collisions
\end{enumerate}

In this lab, you will use collisions between carts to explore the concepts of conservation of momentum and conservation of energy.

\subsubsection*{What you should turn in:} 
An informal report in a single document. You may submit a group report.
\begin{itemize}
\setlength{\parskip}{3pt}
\item Part 1: Responses to qualitative questions. [4 pts]
\item Part 2: Derivations, calculations, and responses to questions. [6 pts]
\item Part 2: Plots of velocity vs. time, momentum vs. time, and kinetic energy vs. time. [6 pts]
\item Part 3: Plots of total momentum vs. time and total kinetic energy vs. time. [6 pts]
\item Part 3: Discussion of changes in momentum and energy during the collisions. [2 pts]
\end{itemize}

\subsubsection*{Equipment}
\begin{itemize}
\setlength{\parskip}{3pt}
\item Carts and tracks
\item LabQuest interfaces
\item Motion sensors
\end{itemize}

\section{QUALITATIVE ANALYSIS OF COLLISIONS}
You will qualitatively observe different types of collisions between two carts to see how their resultant motion is affected by conservation of momentum. The total momentum of a system, $\vec P$, is given by
$$\vec P=\displaystyle\sum_j^N{m_j\vec v_j},$$
where $N$ is the number of objects in the system and $m$ and $\vec v$ are the mass and velocity of each of the objects. Conservation of momentum indicates that the total momentum of the system does not change with time if the next external force on the system is 0. In other words, 
$$\vec P_i = \vec P_f,$$
where $\vec P_i$ is the initial total momentum and $\vec P_f$ is the total momentum at some later time (e.g., after a collision). %For the simple two-object collisions that you will investigate, this means that
%$$m_1 v_{1,i} + m_2 v_{2,i} = m_1 v_{1,f} + m_2 v_{2,f}.$$
%For the \textit{special case} that the mass of the objects is the same (i.e., $m_1=m_2$) we can write
%$$m_1(v_{1,i}+v_{2,i}) = m_1(v_{1,f}+v_{2,f})$$
%and therefore
%$$v_{1,i}+v_{2,i} = v_{1,f}+v_{2,f}.$$ 
All \textit{conservation laws} work this way: the total amount of some quantity (in this case momentum) remains the same.


{\bf 1.1. Elastic collision with equal masses}\\ 
Orient the carts so that their magnets repel each other. Roll one cart toward the other. The target cart is initially at rest. The mass of each cart should be approximately equal. Just from visually observing the collision, what seems to have happened? (Was momentum conserved, and how do you know?)

{\bf 1.2. Mirror symmetry}\\
Now repeat the collision from 1.1, but do everything as a mirror image.  The roles of the target cart and incoming cart are reversed, and the direction of motion is also reversed.  What did you observe during the collision?  

{\bf 1.3. An explosion}\\
Now start with the carts held close together, with their magnets repelling. Make sure to leave a little gap between the carts so that the velcro doesn't stick. As soon as you release them, they'll fly apart due to the repulsion of the magnets.  What do you observe during the ``collision''? Was momentum conserved?

{\bf 1.4. Head-on elastic collision}\\
Now try a collision in which the two carts head towards each other at equal speeds (meaning that one cart's initial velocity is positive, while the other's is negative). Have the magnets point toward each other, so that the carts don't quite collide with each other. What do
you observe during the collision?

{\bf 1.5. Unequal masses}\\ 
Now put a mass on one of the carts and leave
the other cart with no additional mass.  Make the heavier cart hit the
initially stationary cart without additional weight. Does it appear
that momentum is conserved? What if the lighter cart hits the heavier cart? Describe what you observe.

{\bf 1.6. Sticking (perfectly inelastic collisions)}\\ 
Arrange a collision in which the carts will stick together rather than rebounding.  You can do this by letting the velcro ends of the carts hit each other instead of the
magnet ends.  Make a collision in which the target is initially stationary.  What do you observe during the collision?


\section{PERFECTLY INELASTIC COLLISIONS}
% Start the software and initialize the calibration and experiment folders as we have done in previous labs.
A perfectly inelastic collision is one in which two or more objects stick together, such as in part 1.6. This type of collision results in a large drop in the total kinetic energy of the system, $K_{tot}$, which is given by
$$K_{tot} = \sum_j^N \frac{1}{2}m_jv_j^2$$
where $v$ is the speed of each object. Kinetic energy is energy associated with motion.
 
You will now repeat part 1.6 by making quantitative measurements with the LoggerPro software and motion sensor. Using the motion sensor, collect the initial velocity and final velocity of both carts during an inelastic collision. The motion sensors will point in opposite directions. Since velocity is a vector, you will need to reverse the sign of one of your data sets to ensure that you are consistent with how you define positive velocity. Make sure that your track is as level as possible.

\textbf{2.1.} Submit the following graphs:
\begin{itemize}
\item Velocity vs. time: Plot the velocity of both carts on the same graph. 
\item Momentum vs. time: Plot the momentum of both carts as well as the total momentum on the same graph. 
\item Kinetic energy vs. time: Plot the kinetic energy of both carts as well as the total moment kinetic energy on the same graph. 
\end{itemize}

Be sure to label the axes and individual lines.

\textbf{2.2.} In this experiment, the initial velocity of the target cart is $\vec v_2=0$. After the collision, the carts move at the same velocity, so that $\vec v_{1,f}=\vec v_{2,f}$. Use conservation of momentum to derive an equation relating $\vec v_{1,f}$ to $\vec v_{1,i}$. Is this consistent with your observations?

\textbf{2.3.} In class we'll see that the total energy of a system is constant ($\Delta E=0$) if there are no external forces acting on the system. Did the kinetic energy of the system change during this collision (i.e., was it converted to thermal energy)? If so, what percent of the kinetic energy was ``lost''?

\textbf{2.4.} After the collision you should see that the speed of the (combined) carts decreases due to friction. During this deceleration, what is the average rate at which kinetic energy is converted to thermal energy? (HINT: Look at how the kinetic energy is changing with time.)



\section{ELASTIC VS INELASTIC COLLISIONS}
You will now compute the change in total kinetic energy during a collision for two different collisions.  The set-up is similar to part 2. For both collisions, you will give the carts an initial velocity toward each other (try to push them with roughly the same speed) and again you will need to use two motion sensors to record the motion of both carts. 

In the first collision, make sure that one of the carts is a dynamic cart that has a spring (but don't load the spring). The spring should get compressed during the collision. This type of collision is a (general) inelastic collision. 

In the second collision, orient the carts' magnets so that the carts don't actually make contact with each other during the collision. This type of collision is referred to as an elastic collision.

\textbf{3.1.} For both collisions you should produce plots of the total kinetic energy vs. time and the total momentum vs. time. As in part 2, you will need to be careful when calculating the momentum since momentum is a vector (i.e., the sign of the velocity matters) and the motion sensors that you are using to observe the carts are pointing in opposite directions.

\textbf{3.2.} Describe the changes in kinetic energy and momentum during the collisions. What is a defining characteristic of elastic collisions?


\end{document}
