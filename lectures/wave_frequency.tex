\section{Traveling waves, Doppler effect, and shock waves}
Objectives:
\begin{itemize}
\item Review description of traveling waves
\item Doppler effect
\item Shock waves
\end{itemize}

\subsection{Review of traveling waves}
\begin{itemize}
\item Wave model for waves produced by simple harmonic oscillator: $\Delta{y}(x,t)=A\cos\left(2\pi\frac{x}{\lambda}\pm 2\pi\frac{t}{T}\right)$
\item Waves may be transverse or longitudinal
\item Wave speed: $v=\frac{\lambda}{T}=\lambda f$
  \begin{itemize}
  \item Wave on a string: $v=\sqrt{\frac{F_t}{\mu}}$, where $F_t$ is the tensional force and $\mu$ is the linear density (mass / length).
  \item Sound wave in an ideal gas: $v_{sound}=\sqrt{\frac{\gamma RT}{M}}$,  where $\gamma=c_p/c_v$ is the adiabatic index and is often close to one, $R=8.31\mbox{ J/(mol}\cdot\mbox{K)}$ is the gas constant, $T$ is temperature in Kelvin, and $M$ is the molar mass (kg/mol of the gas). The speed of sound basically depends on temperature. For non-ideal gases, $v_{sound}$ has a slight dependency on density, and therefore pressure, but we won't worry about that.
  \item Electromagnetic waves in a vacuum: $v=c\approx 3\times 10^8\mbox{ m/s}=\mbox{constant}$; for travel through a material (e.g., glass or water), $v=\frac{c}{n}$, where $n\geq 1$ is the index of refraction. It essentially has to do with interactions between the electromagnetic wave and electrons in the material.
  \end{itemize}  
\item Note that the wave model describes how the displacement caused by a wave varies in time and space. It does not determine what controls the the wave speed, frequency, wavelength, and amplitude of specific types of waves. Those are things that are unique to a system.
\end{itemize}

For waves that have a constant velocity, if you know the frequency or period you can easily calculate the wavelength, and vice-versa.

high frequency waves = short wavelength = high pitch (sound) or color blue (EM waves)

low frequency waves = long wavelength = low pitch (sound) or color red (EM waves)

[Demo: sound chirp with audacity]

The human ear can detect sounds between about 20 Hz and 20 kHz. Frequencies less than 20 Hz are referred to as infrasound. These waves travel large distances, and are used for detecting nuclear explosions, studying volcanoes and earthquakes, ... Frequencies greater than 20 kHz are referred to as ultrasound. Their short wavelengths make them useful for imaging soft tissue/fetuses.

\subsubsection{Example \#1: Find wave frequency}
A sinusoidal wave travels with speed 200 m/s. Its wavelength is 4.0 m. What is its frequency? 

$$v=\frac{\lambda}{T}=\lambda f$$

$$f = \frac{v}{\lambda} = 50\mbox{ Hz}$$


\subsubsection{Example \#2: Analysis of wave model}
A traveling wave has displacement given by $y(x,t)=(2.0\mbox{ cm})\times\cos(2\pi x-4\pi t)$, where x is measured in cm and t in s.

(a) Draw a snapshot graph of this wave at t=0 s.

(b) On the same set of axes, use a dotted line to show the snapshot graph of the wave at t=1/8 s.

(c) What is the speed of the wave?

[Snapshot graph.]\nopagebreak
\vspace{5cm}

Discuss phase shift using sum and difference identity.

$$v=\frac{\lambda}{T}$$

$$\frac{2\pi}{\lambda}=2\pi \Rightarrow \lambda=1\mbox{ cm}$$
$$\frac{2\pi}{T}=4\pi \Rightarrow T = 2\mbox{ s}$$
$$\boxed{v=0.5\mbox{ cm/s}}$$




\subsubsection{Example \#3: Wave on spider silk}
A female orb spider has a mass of 0.50 g. She is suspended from a tree branch by a 1.1 m length of 0.0020-mm-diameter silk. Spider silk has a density of 1300 kg/m$^3$. If you tap the branch and send a vibration down the thread, how long does it take to reach the spider?

$$v=\sqrt{\frac{F_t}{\mu}}$$

The tensional force is found from Newton's first law
$$\sum F_y = F_t-F_g = 0 \Rightarrow F_t = mg,$$
and the linear density is
$$\mu = \frac{m_s}{L} = \frac{\rho V}{L} = \frac{\rho\pi (d/2)^2L}{L}=\frac{1}{4}\rho\pi d^2$$

Putting this together,
$$v=\sqrt{\frac{4mg}{\rho \pi d^2}} = \frac{2}{d}\sqrt\frac{mg}{\rho \pi}=1.10\times 10^3\mbox{ m/s}$$

The time that it takes to travel a distance $L$ is
$$\Delta t = \frac{L}{v} = 10^{-3}\mbox{ s} = 1.0\mbox{ ms}$$
  
\subsection{Doppler effect}
Many wave sources emit spherical waves (this is difficult to visualize). The sound that is heard depends on whether or not the source is moving. This is referred to as the Doppler Effect.

[Diagram of stationary source and equally-spaced spherical waves.]\nopagebreak
\vspace{5cm}

[Diagram of moving source, showing that frequency depends on position relative to the source.]\nopagebreak
\vspace{5cm}

If the wave is travelling to the right at speed $v_o$, and the source is travelling at $v_s$, the observed wavelength is 
$$\lambda=v_oT-v_sT=(v_o-v_s)T_o=\frac{v_o-v_s}{f_o}.$$
But you perceive the wavelength as 
$$\lambda=\frac{v_o}{f}$$
Setting these equal gives
$$\frac{v_o}{f}=\frac{v_o-v_s}{f_o}$$
Taking the inverse
$$\frac{f}{v_o}=\frac{f_o}{v_o-v_s}$$
Multiplying by $v_o$ gives
$$\boxed{f=\frac{f_o}{1-v_s/v_o}}$$

This is for an object moving toward you. For an object moving away from you, the result is
$$\boxed{f=\frac{f_o}{1+v_s/v_o}}$$

If the object is moving toward you and $v_s=v_0$, the equation blows up. This is when you get a shock wave (``sonic boom'' for sound waves). If $v_s>v_0$ you get a negative frequency, which doesn't make sense.

[Demo: Animations of sound waves illustrating the Doppler effect and shock waves]

[Demo: Whirling tuning fork.]

\subsubsection{Example \#4: Doppler effect on a bicycle}
A whistle you use to call your hunting dog has a frequency of 21 kHz, but your dog is ignoring it. You suspect the whistle may not be working, but you can’t hear sounds above 20 kHz. To test it, you ask a friend to blow the whistle, then you hop on your bicycle. In which direction should you ride (toward or away from your friend) and at what minimum speed to know if the whistle is working?

You need to ride away from your friend to reduce the frequency, and you need to use the equation
$$f=\frac{f_o}{1+v_s/v_o}$$
Solving for $v_s$:
$$v_s = \left(\frac{f_o}{f}-1\right)v_o$$

$v_o=343\mbox{ m/s}$, $f_o=21\mbox{ kHz}$, $f=20\mbox{ kHz}$

$$v_s=17.1\mbox{ m/s}$$

\clearpage
