\section{Traveling waves, Doppler effect, and shock waves}
Objectives:
\begin{itemize}
\item Review description of traveling waves
\item Doppler effect
\item Shock waves
\end{itemize}

\subsection{Review of traveling waves}
\begin{itemize}
\item Wave model for waves produced by simple harmonic oscillator: $\Delta{y}(x,t)=A\cos\left(2\pi\frac{x}{\lambda}\pm 2\pi\frac{t}{T}\right)=A\cos\left(2\pi\frac{x}{\lambda}\pm 2\pi ft\right)$
\item Waves may be transverse or longitudinal
\item Wave speed: $v=\frac{\lambda}{T}=\lambda f$
  \begin{itemize}
  \item Wave on a string: $v=\sqrt{\frac{F_t}{\mu}}$, where $F_t$ is the tensional force and $\mu$ is the linear density (mass / length).
  \item Sound wave in an ideal gas: $v_{sound}=\sqrt{\frac{\gamma RT}{M}}$,  where $\gamma=c_p/c_v$ is the adiabatic index and is often close to one, $R=8.31\mbox{ J/(mol}\cdot\mbox{K)}$ is the gas constant, $T$ is temperature in Kelvin, and $M$ is the molar mass (kg/mol of the gas). The speed of sound basically depends on temperature. For non-ideal gases, $v_{sound}$ has a slight dependency on density, and therefore pressure, but we won't worry about that.
  \item Electromagnetic waves in a vacuum: $v=c\approx 3\times 10^8\mbox{ m/s}=\mbox{constant}$; for travel through a material (e.g., glass or water), $v=\frac{c}{n}$, where $n\geq 1$ is the index of refraction. It essentially has to do with interactions between the electromagnetic wave and electrons in the material.
  \end{itemize}  
\item Note that the wave model describes how the displacement caused by a wave varies in time and space. It does not determine what controls the the wave speed, frequency, wavelength, and amplitude of specific types of waves. Those are things that are unique to a system.
\end{itemize}

For waves that have a constant velocity, if you know the frequency or period you can easily calculate the wavelength, and vice-versa.

high frequency waves = short wavelength = high pitch (sound) or color blue (EM waves)

low frequency waves = long wavelength = low pitch (sound) or color red (EM waves)

[Demo: sound chirp with audacity]

The human ear can detect sounds between about 20 Hz and 20 kHz. Frequencies less than 20 Hz are referred to as infrasound. These waves travel large distances, and are used for detecting nuclear explosions, studying volcanoes and earthquakes, ... Frequencies greater than 20 kHz are referred to as ultrasound. Their short wavelengths make them useful for imaging soft tissue/fetuses.

\subsubsection{Example problems}
\subsubsection*{Example \#1: Find wave frequency}
A sinusoidal wave travels with speed 200 m/s. Its wavelength is 4.0 m. What is its frequency? 

$$v=\frac{\lambda}{T}=\lambda f$$

$$f = \frac{v}{\lambda} = 50\mbox{ Hz}$$


\subsubsection*{Example \#2: Analysis of wave model}
The displacement of traveling longitudinal wave is given by
$$\Delta{x} = 5.2\cos(5.5x + 72t),$$
where $\Delta {x}$ is in cm and $x$ is in m. What are the wavelength, frequency, speed, and travel direction of the wave?

Solving this is simply a matter of comparing the equation to the wave model. From that, we see:

$$2\pi \frac{x}{\lambda} = 5.5x$$
which means that
$$\lambda = \frac{2\pi}{5.5} = 1.14\mbox{ m}$$

$$2\pi ft = 72t$$
which means that
$$f = \frac{72}{2\pi} = 11.5\mbox{ Hz}$$

The wave speed is $v=\lambda f = 13.1\mbox{ m/s}$. The plus sign indicates that the wave is traveling in the negative x-direction.



\subsubsection*{Example \#3: Wave on spider silk}
A female orb spider has a mass of 0.50 g. She is suspended from a tree branch by a 1.1 m length of 0.0020-mm-diameter silk. Spider silk has a density of 1300 kg/m$^3$. If you tap the branch and send a vibration down the thread, how long does it take to reach the spider?

$$v=\sqrt{\frac{F_t}{\mu}}$$

The tensional force is found from Newton's first law
$$\sum F_y = F_t-F_g = 0 \Rightarrow F_t = mg,$$
and the linear density is
$$\mu = \frac{m_s}{L} = \frac{\rho V}{L} = \frac{\rho\pi (d/2)^2L}{L}=\frac{1}{4}\rho\pi d^2$$

Putting this together,
$$v=\sqrt{\frac{4mg}{\rho \pi d^2}} = \frac{2}{d}\sqrt\frac{mg}{\rho \pi}=1.10\times 10^3\mbox{ m/s}$$

The time that it takes to travel a distance $L$ is
$$\Delta t = \frac{L}{v} = 10^{-3}\mbox{ s} = 1.0\mbox{ ms}$$
  
\subsection{Doppler effect for moving sources and stationary observers}
Many wave sources emit spherical waves (this is difficult to visualize). The sound that is heard depends on whether or not the source is moving. This is referred to as the Doppler Effect.

[Diagram of stationary source and equally-spaced spherical waves.]\nopagebreak
\vspace{5cm}

[Diagram of moving source, showing that frequency depends on position relative to the source.]\nopagebreak
\vspace{5cm}

If the wave is traveling to the right at speed $v$, and the source is travelling at $v_s$, the observed wavelength is 
$$\lambda_o=vT-v_sT=(v-v_s)T_o=\frac{v-v_s}{f}.$$
But you perceive/observe the wavelength as 
$$\lambda_o=\frac{v}{f_o}$$
Setting these equal gives
$$\frac{v}{f_o}=\frac{v-v_s}{f}$$
Taking the inverse
$$\frac{f_o}{v}=\frac{f}{v-v_s}$$
Multiplying by $v$ gives
$$\boxed{f_o=\left(\frac{v}{v-v_s}\right)f, \mbox{ source approaching}}$$

This is for an object moving toward you. For an object moving away from you, the result is
$$\boxed{f_o=\left(\frac{v}{v+v_s}\right)f, \mbox{ source receding}}$$

If the object is moving toward you and $v_s=v$, the equation blows up. This is when you get a shock wave (``sonic boom'' for sound waves). Shock waves occure if $v_s\geq v$.

[Demo: Animations of sound waves illustrating the Doppler effect and shock waves]

[Demo: Whirling tuning fork.]

\subsubsection*{Example \#4: Doppler effect on a bicycle}
A whistle you use to call your hunting dog has a frequency of 21 kHz, but your dog is ignoring it. You suspect the whistle may not be working, but you can’t hear sounds above 20 kHz. To test it, you ask a friend to blow the whistle while riding a bicycle. In which direction should they ride (toward or away from you) and at what minimum speed to know if the whistle is working?

You need to ride them to ride away from you to reduce the frequency, and you need to use the equation
$$f_o=\left(\frac{v}{v+v_s}\right)f$$
Solving for $v_s$:
$$v_s = \left(\frac{f}{f_o}-1\right)v$$

$v=343\mbox{ m/s}$, $f=21\mbox{ kHz}$, $f_o=20\mbox{ kHz}$

$$\boxed{v_s=17.2\mbox{ m/s}}$$

What if you blew the whistle while your friend bike toward or away from you? You might think it's the same result, but it's not. In the first case, the movement of the source is changing the physical character of the waves, whereas in the second the movement is changing the rate at which the person encounters the waves. We need to use a different equation (see next section):

$$f_o = \left(\frac{v+v_o}{v}\right)f$$
Solving for $v_o$,
$$v_o = v\left(1-\frac{f_o}{f}\right)=\boxed{16.3\mbox{ m/s}}$$




\subsection{Doppler effect for stationary sources and moving observers}
If you are traveling toward the source, the time that it takes to pass from the first wavefront to the second wavefront is $T_o$. The distance that you travel is
$$v_oT_o = \lambda - vT_o \Rightarrow (v+v_o)T_o = \lambda$$
We also know that the wavelength is
$$\lambda = vT$$
Setting these equations equal to each other,
$$(v+v_o)T_o=vT\rightarrow \boxed{f_o = \left(\frac{v+v_o}{v}\right)f,\mbox{ moving toward source}}$$

You can use a similar to analysis to show that
$$\boxed{f_o = \left(\frac{v-v_o}{v}\right)f, \mbox{ moving away from source}}$$



\clearpage
