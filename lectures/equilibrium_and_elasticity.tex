\section{Equilibrium and elasticity}
Objectives:
\begin{itemize}
\item Stability
\item Elastic deformation
\end{itemize}

Last class we discussed problems involving static equilibrium. When an object is in static equilibrium,
$$\sum F_x=0$$
$$\sum F_y=0$$
$$\sum \tau=0$$
When calculating torques, you are free to pick any axis of rotation (the book refers to this as a ``pivot point'') that you'd like. Torque can be caused by any force, including gravity. 

\subsection{Stability}
Gravitational torque is especially important in the context of stability and balance. The gravitational torque is calculated by applying the gravitational force at the object's center of mass. OK, but what is meant by an object's center of mass?

Center of mass is the ``unique point where the weighted relative position of the distributed mass sums to zero''. There is a gravitational force acting downward on each point within an object. Adding up the torques produced by each of these forces is equivalent to applying the total gravitational force to the object's center of mass.

We can find the angle at which an object will tip over by calculating the torque as a function of the tilt angle. When $\tau=0$ the object is either completely stable or meta-stable. If it is meta-stable, tilting it a little bit one way or the other will cause it to topple.

The threshold for determining which direction an object will tip occurs when $\sum \tau =0$. When we think about torque due to gravity, keep in mind that gravity acts on an object's center of mass. 

[Insert diagrams of boxes tipping over.]
\vspace{5cm}

\clearpage
Consider the stability of a car.

[Insert diagram of a car with width $W$ and center-of-mass height $H$.]
\vspace{5cm}

When the car is just about to roll, the center of mass of the car is positioned directly above the wheels. In other words, the torque acting about that point is zero. The critical angle, $\theta_c$, at which this happens is related to $W$ and $H$:
$$\tan\theta_c=\frac{W/2}{H}\Rightarrow \theta_c=\tan\ds^{-1}\frac{W}{2H}$$

With a wider or shorter car, you have to tilt the car to a larger angle before it rolls over.

A related exercise is the ladder leaning against the wall (see previous lecture notes).


\subsection{Elasticity}
So far in our analysis of static equilibrium we've assumed that objects maintain their shape. In reality, all objects stretch, compress, and deform. We've already talked about this a little bit in the context of normal forces and tensional forces, which are a result of compression and tension occurring at the molecular scale. These forces are ``restoring forces'' that try to restore the objects to their equilibrium position. Materials that have restoring forces are called ``elastic''. 

We can model elasticity of many materials using Hooke's Law. Recall that
$$F_{sp}=-k\Delta{x}$$
where $k$ is a spring constant that is unique to a spring. We will replace $k$ with
$$k=\frac{YA}{L}$$
where $Y$ is Young's modulus (a material property), $A$ is the cross-sectional area of the object, and $L$ is the length of the object. An object with a large Young's modulus is resistant to changes in length.

When writing the elastic force, we often replace $\Delta{x}$ with $\Delta{L}$. So
$$F=-\frac{YA}{L}\Delta{L}\Rightarrow \frac{F}{A}=-Y\frac{\Delta{L}}{L}$$
$F/A$ is the stress and $\Delta{L}/L$ is the strain (i.e., fractional change in length).

What does this equation tell us? What if we apply a large force to a thin rope? What if we construct a thicker rope out of the same material?

This equation also tells us that we should think about normal forces (and tensional forces) as being applied everywhere that two surfaces are in contact. However, instead of adding up a whole bunch of little forces, it is much easier to calculate the total force and apply it at the center of where two object's come in contact.

If we were to make a plot of $F/A$ vs. $\Delta{L}/L$, we would find a linear relationship for small $\Delta{L}$. This is the range for which Hooke's Law is valid. Beyond that the material will remain elastic for a little bit more deformation until it reaches the elastic limit. Beyond that the material behaves plastically --- it changes length will very little change in force --- until it reaches the breaking point, or tensile/compressive strength.

[Insert diagram.]
\vspace{5cm}

Interestingly, some materials stretch a lot before breaking while others can only stretch a little bit. For example, steel and spider silk have similar tensile strengths (break at the same stress), but silk undergoes much more extension than steel.

[Insert diagram.]
\vspace{5cm}

Like torque, elasticity is also a very important concept in biology --- for example, it comes up in the context of extension and compression of bones.

\subsection{Example \#1: Elevator cable stretch}
A 2300~kg car of a high-speed elevator is supported by six 1.27-cm-diameter cables. Young's moduls for the cables is $10\time10^{10}$~N/m\textsuperscript{2}. When the elevator is on the bottom floor, the cables rise 90~m up the shaft to the motor.

On a busy morning, the elevator is on the bottom floor and fills up with 20 people who have a total mass of 1500~kg. The elevator accelerates upward at 2.3~m/s\textsuperscript{2} until it reaches the cruising speed. (1) How much do the cables stretch due to the weight of the car? (2) How much additional stretch occur when the passengers are in the car? (3) And what is the total stretch of the cables when the elevator is accelerating? (In all cases, ignore the weight of the cables.)

[Insert diagram of elevator.]
\vspace{4cm}

(0) Stress-strain relationship
$$\frac{F}{A} = Y\frac{\Delta L}{L}$$
$$\Delta L = \frac{FL}{YA}$$
The length is $L=90$~m.

The area is six times the area of each cable
$$A = 6\pi r^2 = 7.62\times 10^{-4}\mbox{ m}^2$$

(1) Stretch due to car
$$\sum F = F_t - F_g = 0$$
$$F_t = m_{\mathrm car}g = 22500\mbox{ N}$$

$$\boxed{\Delta L = 2.7\mbox{ cm}}$$

(2) Stretch due to car and passengers
$$\sum F = F_t - F_g = 0$$
$$F_t = m_{\mathrm car}g = 37200\mbox{ N}$$

$$\boxed{\Delta L = 4.4\mbox{ cm}}$$

The cables stretch by 1.7 cm when the passengers board the elevator.

(3) Elevator accelerating upward
$$\sum F = F_t - F_g = ma$$
$$F_t = m(g+a) = 46000\mbox{ N}$$

$$\boxed{\Delta L = 5.4\mbox{ cm}}$$

The cables stretch by an additional 1~cm.


\subsection{Example \#2: Plank supported by a rope}
A 100-kg, 3.5-m-long plank is supported on its end by a 7.0-mm-diameter rope with a tensile strength of 6.0$\times$10$^7$ N/m$^2$. How far along the plank, measured from the pivot, can an 800-kg piece of machinery be moved along the plank before the rope snaps?

[Insert diagram.]
\vspace{5cm}

We want to find at what point does $F_t/A=6.0\times 10^7\mbox{ N/m}^2$. The cross-sectional area is $A=\pi r^2=3.85\times 10^{-5}\mbox{ m}^2$, so we need to determine when $F_t=2310\mbox{ N}$.

$$\sum F_y=F_n-F_{g,p}-F_{g,m}+F_t=0$$
$F_t$ and $F_n$ are unknown, so we need one another equation. We want to remove $F_n$ from the first equation. The way to do this is to sum the torques around the point where the rope is holding up the plank. Let's call the length of the blank $L$, and the distance from the pivot to the machinery $l$.
$$\sum \tau=F_{g,p}\frac{L}{2}+F_{g,m}(L-l)-F_nL=0$$
$$F_nL=F_{g,p}\frac{L}{2}+F_{g,m}(L-l)$$
$$F_n=\frac{1}{2}F_{g,p}+F_{g,m}\left(1-\frac{l}{L}\right)$$
Let's plug the result into the force balance equation.
$$\frac{1}{2}F_{g,p}+F_{g,m}\left(1-\frac{l}{L}\right)-F_{g,p}-F_{g,m}+F_t=0$$
$$-\frac{1}{2}F_{g,p}-F_{g,m}\frac{l}{L}+F_t=0$$
Solve this for $l$; we want to find for what $l$ does $F_t=2310\mbox{ N}$.
$$F_{g,m}\frac{l}{L}=F_t-\frac{1}{2}F_{g,p}$$
$$l=\left(F_t-\frac{1}{2}F_{g,p}\right)\frac{L}{F_{g,m}}$$
Now we can plug in values, and we find that
$$\boxed{l=0.81\mbox{ m}}$$

\clearpage
