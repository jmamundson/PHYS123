\section{ROTATIONAL MOTION}
Last class I introduced several terms used to describe circular motion (e.g., a satellite revolving around the Earth).
\begin{itemize}
\itemsep 0pt
\item Angular position, $\theta$ [rad], measured counterclockwise from positive $x-$axis
\item Angular displacement, $\Delta\theta$ [rad]
\item Angular velocity, $\omega=\frac{\Delta\theta}{\Delta{t}}$ [rad/s]
\item Relationship between frequency and angular velocity, $\omega=2\pi f =2\pi/T$
\item Angular acceleration, $\alpha=\frac{\Delta\omega}{\Delta{t}}$ [rad/s$^2$]
\item Tangential velocity, $v_t=\omega r$ [m/s]
\item Tangential acceleration, $a_t=\alpha r$ [m/s$^2$] (didn't show last time)
\item Centripetal acceleration, $a_c=\frac{v_t^2}{r}=\omega^2r$ [m/s$^2$]
\item We can use essentially the same kinematic equations as before to describe circular motion, we just need to replace linear quantities with angular quantities.
\end{itemize}

(Start by reviewing derivation of centripetal acceleration; then work on a couple of example problems from last class.)
\vspace{5cm}

\subsection*{Introduction}
Everything that we've learned so far about circular motion also applies to rotational motion (i.e., the rigid body rotation of an object around some axis). An example of rotational motion is a bicycle wheel spinning around its axle. The fundamental difference between circular motion and rotation motion is that in rotational motion, all parts of an object DO NOT move at the same \textit{linear} speed/velocity. They do have the same angular velocity.

And all of the equations for circular motion can be easily adapted to rotational motion. So we know that the tangential speed is $v_t=\omega{r}$. The parts of the bicycle wheel that are farthest from axle move the fastest. 

If we want a fast bicycle, should we make the wheels big or small? (Actually its kind of a trick question, because big wheels are more difficult to turn!)

\subsection*{Example: Rope around axle of a cart}
You wrap a rope around the axle of a cart. The axle is 8 cm in diameter, and the wheels on the cart are 1 m in diameter. Assume that there is perfect friction between the rope and axle; in other words, the wheels roll when you pull on the rope without slipping. If you pull the rope at 0.5 m/s, how quickly will the cart move (toward you!)? 

Given:\\
$r_{axle}=0.04\mbox{ m}$\\
$r_{wheel}=0.5\mbox{ m}$\\
$v_{t,axle}=1\mbox{ m/s}$

Want to know: $v$

How do we solve this? Let's first think about rolling motion.

[Insert diagram showing trajectory of a particle on the outside of the wheel.]
\vspace{5cm}

If the wheel rotates without slipping, during one revolution the center of the wheel will have moved forward a distance
$$\Delta x=v\Delta t=2\pi R,$$
and so
$$v=\frac{\Delta x}{\Delta t}=\frac{2\pi R}{\Delta t}.$$
Since the time to turn one revolution is the period, $T$, we find that
$$v=\frac{2\pi R}{T}.$$
Can we further simplify? Yes, we saw earlier that $\omega=2\pi/T$, so
$$\boxed{v=\omega R}$$
This is referred to as the rolling constraint.

From this analysis we can also deduce that rolling motion is a combination of translation and rotation.

\clearpage
[Insert diagram of translation + rotation = rolling.]
\vspace{5cm}

Back to our example problem: once we calculate the angular velocity of the wheel, it is straightforward to calculate its speed.
$$v_{axle}=\omega r_{axle}$$
$$\omega=\frac{v_{t,axle}}{r_{axle}}=25\mbox{ rad/s}$$
$$v=\omega R=12.5\mbox{ m/s}$$
This is slightly faster than a person can run.

\clearpage
