\section{Fluid dynamics II}
Objectives:
\begin{itemize}
  \item Apply ideas from last class
    \begin{enumerate}
    \item Ideal fluid: incompressible, laminar, inviscid
    \item Flux continuity: $Q=vA=\mbox{constant}$
    \item Bernoulli's equation: $P+\frac{1}{2}\rho v^2+\rho gh=\mbox{constant}$
    \end{enumerate}
  \item Viscosity
\end{itemize}


\subsection{Demo using Venturi tube}
One really useful application of Bernoulli's equation is that it can be used to measure fluid flow through pipes.

[Insert diagram.]
\vspace{5cm}

If we look at the streamline that passes through the center of this tube,
$$P_1+\frac{1}{2}\rho v_1\ds^2+\rho gy_1=P_2+\frac{1}{2}\rho v_2\ds^2+\rho gy_2$$
But points 1 and 2 are at the same elevation, so $y_1=y_2$, and so
$$P_1+\frac{1}{2}\rho v_1\ds^2=P_2+\frac{1}{2}\rho v_2\ds^2$$
We can re-arrange this so that
$$P_2-P_1=\frac{1}{2}\rho(v_2\ds^2-v_1\ds^2)$$

From flux continuity, 
$$v_1A_1=v_2A_2$$
so 
$$v_2=v_1\left(\frac{A_1}{A_2}\right)$$

Plugging this into Bernoulli's equation gives
$$P_2-P_1=\frac{1}{2}\rho\left(v_1\ds^2-\left(\frac{A_1}{A_2}\right)^2v_1\ds^2\right)=\frac{1}{2}\rho v_1\ds^2\left(1-\left(\frac{A_1}{A_2}\right)^2\right)$$

What is $P_2-P_1$? The air pressure doesn't change much with elevation (in part because there is no flow in the vertical direction). This means that
$$P_1+\rho gh=P_2$$
where $h$ is the difference in height of the columns of water. Rearranging,
$$P_2-P_1=\rho gh$$
Inserting this into Bernoulli's equation,
$$\rho gh=\frac{1}{2}\rho v_2\ds^2\left(1-\left(\frac{A_2}{A_1}\right)^2\right)$$

Dividing by $\rho$ and solving for $v_2$ gives
$$\boxed{v_1=\sqrt{\frac{2gh}{1-\left(\frac{A_1}{A_2}\right)^2}}}$$

For this problem, $h=0.1$~m, $\rho=1000\mbox{ kg/m}^3$, $r_1=0.01$~m, $r_2=0.03$~m, and therefore $A_1/A_2=1/9$.

And this means that
$$\boxed{v_1=1.4\mbox{ m/s}}$$

What is the flux?
$$Q=v_1A_1 = v_1\pi r_1^2 = 0.00044\mbox{ m}^3\mbox{/s}$$

Human lungs hold about 6 L of air, or 0.006~m$^3$. With this flux, it would take
13.5~s to empty your lungs ($\Delta t = V_{lung}/Q$).

\subsection{Viscosity}
Let's see what happens if we relax the condition that the fluid is inviscid, but maintain the other conditions.

[Insert diagram of fluid flow through a pipe with friction.]
\vspace{5cm}


From conservation of energy:
$$W=\Delta{K}+\Delta{U_g}+\Delta{E_{th}}$$
Following the same steps as when we derived Bernoulli's equation,
$$(P_1-P_2)\Delta{V}=\frac{1}{2}\rho\Delta V(v_2\ds^2-v_1\ds^2)+\rho g(h_2-h_1)+\Delta{E_{th}}$$
If we consider a horizontal pipe with uniform diameter, $h_1=h_2$ and flux continuity requires that $v_1=v_2$. Therefore,
$$\Delta{E_{th}}=(P_1-P_2)\Delta{V}$$

Friction within the fluid, and between the fluid and walls of the pipe, cause $\Delta E_{th}>0$. This means that $P_2<P_1$.

Experiments show that the pressure difference needed to drive laminar flow of a viscous fluid through a pipe is
$$P_1-P_2=8\pi\eta\frac{Lv_{avg}}{A}$$
where $\eta$ is the fluid viscosity and has units of Pa$\cdot$s. The viscosity of water is about 10$^{-3}$ Pa$\cdot$s, while the viscosity of honey is 20--600 Pa$\cdot$s. This equation is referred to as Poiseuille's Equation. It tells us that the average velocity of fluid flowing through a section of pipe is 
$$v_{avg}=\frac{(P_1-P_2)}{L}\frac{A}{8\pi\eta}$$
In other words, you need a pressure gradient to drive flow, large pipes provide less resistance, and fluids with high viscosity flow slowly. How do you keep the velocity high?

Examples:
\begin{enumerate}
\item Alaska pipeline: Use pump stations to create large pressure gradients.
\item Clogged arteries: Use blood thinner to reduce $\eta$, or scraping out arteries to increase $A$.
\end{enumerate}

We can also relate this to the amount of thermal energy that is produced in a section of fluid.
$$\Delta E_{th}=8\pi\eta\frac{Lv_{avg}}{A}\Delta{V}$$
where $\Delta{V}=A\Delta x$, so
$$\Delta E_{th}=(P_1-P_2)\Delta V = 8\pi\eta Lv_{avg}\Delta{x}$$
The amount of thermal energy that is generated in the fluid is proportional to the viscosity, length of pipe, velocity, and displacement of the fluid.

So I've now just opened a can of worms... Viscosity is generally highly temperature dependent, and density is sometimes temperature dependent (so assumption of incompressibility is questionable).

\subsubsection{Example \#1: Application of Poiseuille's equation}
A stiff, 10-cm long tube with an inner diameter of 3.0 mm is attached to a small hole in the side of a tall beaker. The tube sticks out horizontally and is open at one end. The beaker is filled with with 20$^\circ$C water ($\eta=10^{-3}\mbox{ Pa}\cdot\mbox{s}$) to a level 45 cm above the hole, and is continually topped off to maintain that level. What is the volume flow rate (flux) through the tube?

The pressure difference between the ends of the tube is
$$P_1-P_2=(P_{atm}+\rho gh)-(P_{atm})=\rho gh$$

Poiseuille's equation tells us how the pressure difference relates to the flow of the water.
$$P_1-P_2=8\pi\eta\frac{v_{avg}L}{A}=\rho gh$$
Solving for $v_{avg}$ 
$$v_{avg}=\frac{\rho ghA}{8\pi\eta L}=$$
and multiplying by $A$:
$$Q=\frac{\rho ghA^2}{8\pi\eta L}=\frac{\rho gh\pi^2r^4}{8\pi\eta L}\Rightarrow\boxed{Q=8.8\times 10^{-5}\mbox{ m}^3/\mbox{s}}$$



\clearpage
