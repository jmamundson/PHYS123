\section{Torque I}
Objectives:
\begin{itemize}
\item Torque
\item Moment of inertia
\item Newton's 2\textsuperscript{nd} Law
\end{itemize}

\subsection{Background}
We previously saw how forces can produce circular motion. Essentially, you need a force that produces a centripetal acceleration. This force can be any type of force or combination of forces that points toward the center of a circle.
$$\sum F_r=\frac{mv^2}{r}=m\omega^2r$$

Forces can also cause rigid bodies to rotate around some axis of rotation. We will use the term \textit{torque} to refer to the ability of a force to cause a rotation. We will define torque as
$$\boxed{\tau=rF_\perp}$$
where $\tau$ is torque and has units of N$\cdot$m, $r$ is the radial distance from the axis of rotation to the point at which the force is being applied, and $F_\perp$ is the component of the force that is perpendicular to $r$. Following standard conventions, $\tau>0$ for counter-clockwise rotation. You can also add torques in much the same way as you add forces. This definition of torque will become more clear when we discuss conservation of energy.


\subsection{Example \#1: You use a wrench to loosen a bolt} 
[Diagram of wrench and forces.]
\vspace{5cm}

If you apply force $F$ at a distance $r$ from the center of the bolt, and $r\perp F$, then you are exerting a torque $\tau=rF>0$.  If you exert the force at an angle $\theta=45^\circ$ relative to the radial axis, then $\tau=rF\sin\theta$ (in this case, $r\perp F\sin\theta$). If you double the radius, then $\tau=2rF$.

\subsection{Example \#2: Two forces act on a rod}
Two forces act on a rod that pivots about its center. Force $F_2$ is perpendicular to the rod and is a distance $l$ from the axis of rotation. Force $F_1$ has a magnitude of 20 N and is oriented at a 45$^\circ$ angle to the rod. The force is acting on the other side of the axis of rotation and at a distance of $2l$ from the axis. What should $F_2$ be so that $\tau_{net}=0$?
\clearpage
[Include diagram.]
\vspace{5cm}

$$\sum\tau=2lF_1\sin\theta-lF_1=0$$
$$F_2=2F_1\sin\theta=28\mbox{ N}$$

\subsection{Example \#3: Rank forces in order of smallest to largest torque}
Various forces act on a rod. The rod can pivot around one of its ends. Rank the forces in order of smallest to largest torque.

[Insert diagram.]

\vspace{5cm}

\subsection{Torque and rotational motion}
The real reason to discuss torque (now) is that it affects the rotational motion of an object. To see how, let's start with a really simple system consisting of a point mass, $m$, attached to a massless rod of length $l$. The rod can rotate around its end.
\clearpage
[Insert diagram.]
\vspace{5cm}

A force is exerted on the mass; the force is perpendicular to the rod. It therefore produces a tangential acceleration of the mass.
$$F_\perp=ma_t$$
Recalling that $a_t=\alpha r$,
$$F_\perp=m\alpha r.$$
Since $\tau=rF_\perp$,
$$F_\perp=\frac{\tau}{r}=m\alpha r\Rightarrow \boxed{\tau=mr^2\alpha}$$

What about a system of particles, or a solid object?

[Insert diagram showing forces acting a rotating object.]
\vspace{5cm}

From this we have $\tau_1=r_1F_1=m_1r_1\ds^2\alpha$, $\tau_2=m_2r_2\ds^2\alpha$, $...$ Because this is rigid body rotation, all points on the body have the same angular acceleration. The net torque is $\tau_{net}=\tau_1+\tau_2+\tau_3+...$
$$\tau_{net}=\alpha\sum_{i=1}^Nm_ir_i\ds^2$$
The summation can only be calculated analytically for simple objects. So instead we replace the summation with
$$I=\sum_{i=1}^Nm_ir_i\ds^2,$$
where $I$ is the moment of inertia, giving
$$\tau_{net}=I\alpha$$
or equivalently,
$$\boxed{\sum\tau=I\alpha}$$
Does this look at all familiar? It is Newton's Second Law for rotation. The moment of inertia is a rotational equivalent to mass; it is an object's resistance to rotation. The moment of inertia can be calculated for some objects, for others it must be measured. Some that can be calculated:

Solid cylinder: $I=\frac{1}{2}MR^2$\\
Thin-walled cylinder: $I=MR^2$\\
Solid sphere: $I=\frac{2}{5}MR^2$\\
Spherical shell: $I=\frac{2}{3}MR^2$

Note that they all have a dependence on mass and radius squared.

\subsection{Example \#4: torque on rotating disk}
A 0.2 kg, 0.2-m-diameter disk is spun around its central axis by a motor. What torque must the motor supply to take the disk from 0 to 1800 rpm in 4.0 s?

$$\tau_{net}=\sum\tau=I\alpha$$
$$\alpha=\frac{\Delta\omega}{\Delta t}$$

Given:\\
$\omega_i=0$\\
$\omega_f=1800\mbox{ rpm}\times\frac{1\mbox{ min}}{60\mbox{ s}}\times\frac{2\pi}{1\mbox{ rev}}=188.5\mbox{ rad/s}$\\
$\Delta t=4.0\mbox{ s}$

So, from this $\boxed{\alpha=47.1\mbox{ rad/s}^2}$.

The moment of inertia can be calculated for a disk: $I=\frac{1}{2}mr^2=1\times 10^{-3}\mbox{ kg}\cdot\mbox{m}^2$.

Inserting into the expression for $\tau$, we get that $\boxed{\tau_{net}=0.05\mbox{ N}\cdot\mbox{m}}$.


\clearpage
