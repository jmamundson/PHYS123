\section{Introduction to kinematics}
Objectives:
\begin{enumerate}
\item Course introduction
\item Key terminology
\item Graphing motion
\end{enumerate}

\subsection{Overview}
We'll start the semester by learning about kinematics. Kinematics is the description of motion, and has the same origins as the word cinema. In a few weeks we'll discuss dynamics, or causes of motion. When we combine kinematics with dynamics, we are studying mechanics. In introductory physics, we'll cover classical mechanics --- one of the oldest branches of science.

Classical mechanics: large, slow moving objects\\
Quantum mechanics: small objects\\
General relativity: fast moving objects\\

[Insert diagram of physics disciplines.]
\vspace{4cm}

\subsection{Types of motion}
motion: change in object's position with time

trajectory: path along which an object travels

For now, we'll only worry about rigid body motion (i.e., no deformation).

Four special cases of motion:

\begin{enumerate}[itemsep=2cm]
\item straight line
\item projectile motion (influenced by gravity)\\ \\
\item circular motion (e.g., planets, satellites)
\item rotational motion\\ \\ \\
\end{enumerate}


\subsection{Representing motion}
Motion diagrams are used to represent where an object is at different points in time.

[Insert diagram of person running.]
\vspace{4cm}

For motion that does not involve deformation or rotation, we can use a \textit{particle model}. Models are simplifications of reality that allow us to focus on the important physics. In a particle model, we assume that all of the mass of an object is focused at a single point and that the all parts of the object move in the same direction with the same speed. In class I'll often use particles or boxes to represent much more complicated objects.

[Insert example particle model.]
\vspace{2cm}

More commonly, we will use graphs to describe the motion of an object.

[Insert example graph.]
\vspace{4cm}

You will need to become comfortable going back and forth between motion diagrams and position-time, velocity-time, and acceleration-time graphs.


\subsection{Terms used to describe motion}
\begin{itemize}
\item position = location of an object

[Insert 1-D diagram; where is the object on this line?]
\vspace{4cm}

position can be positive \textit{or} negative -- this is important\\
units = [L]

We represent motion with a position-time graph.
[Insert position-time graph.]
\vspace{4cm}


\item displacement = change in position, [L] 
$$\Delta{x}=x_f-x_i$$
Can be positive \textit{or} negative (give example).

\item velocity = rate of change of position, [L]/[T]
  $$v=\frac{\Delta{x}}{\Delta{t}}=\frac{x_f-x_i}{t_f-t_i}$$
  (Technically, this is the average velocity during some time interval $\Delta t$. We'll come back to this.)
Velocity can ALSO be positive \textit{or} negative. What does the sign tell you?

Velocity can be represented with a velocity-time graph.
\clearpage
[Insert velocity-time graph, based on position-time graph.]
\vspace{4cm}



Note that we can go from velocity to displacement (i.e., \textit{change} in position) by rearranging the equation for velocity: 
$$\Delta x=v\Delta{t}$$
which is the same thing as calculating the area under the velocity-time graph.

[Insert diagram.]
\vspace{4cm}

Average velocity during the first interval:
$$v_1=\frac{x_1-x_0}{t_1-t_0}\Rightarrow\mbox{ slope of a straight line}$$
$$v_1>0$$

Average velocity during the second interval:
$$v_2=\frac{x_2-x_1}{t_2-t_1}<0$$

\item speed = magnitude of velocity, [L]/[T], always positive

\item instantaneous velocity (i.e., the slope of the line that is tangent to the velocity curve). For those of you that have taken calculus, this is a derivative.

In practice, data is not continuous; velocity calculated from data is an average over some time interval.



\item acceleration = rate of change of velocity, [L]/[T$^2$]
$$a_{avg}=\frac{v_2-v_1}{t_2-t_1}=\frac{\Delta{v}}{\Delta{t}}$$
  Technically, this is the average acceleration over the time period $\Delta t$. As with velocity, we are often interested in the instantaneous acceleration, which is the slope of the line that is tangent to the velocity curve.

Acceleration is represented with an acceleration-time graph.

[Insert acceleration-time graph.]
\vspace{4cm}
  
Not surprisingly, given what we just saw with displacement, it is easy to go from acceleration to a change in velocity:
$$\Delta{v}=a\Delta{t}$$

Acceleration can also be positive \textit{or} negative. What does the sign mean?

Simple rule for determining whether an object is accelerating or decelerating:\\
If $a\cdot v>0$ the object is accelerating, if $a\cdot v<0$ the object is decelerating  
\end{itemize}


\subsubsection{Example \#1: Going from position to acceleration}
The position of a particle is given in the position-time graph below. Draw the corresponding velocity-time and acceleration-time graphs.


\clearpage
