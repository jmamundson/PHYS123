\section{Standing waves and music}
Objectives:
\begin{itemize}
\item Standing waves on strings
\item Standing waves in open-open and open-closed tubes
\item Musical notes  
\end{itemize}

\subsection{Wave superposition}
Last class we saw that waves can constructively or destructively interfere

 Wave superposition:
$$\Delta y(x,t)=\Delta y_1(x,t)+\Delta y_2(x,t)$$
where
  $$\Delta y_i(x,t)=A_i\cos\left(2\pi\frac{x}{\lambda_i}\pm 2\pi\frac{t}{T_i}\right)$$

\subsection{Standing waves}  
Standing waves are waves in which the peaks don't travel. Places with no displacement are called nodes, and places of maximum displacement are referred to as anti-nodes. Standing waves can be thought of as two waves of equal frequency, amplitude, and speed, travelling in opposite directions. Using wave superposition, we showed that the equation for a standing wave on a string that is fixed on both ends is:
$$\Delta y(x,t)=A\sin\left(2\pi\frac{x}{\lambda}\right)\sin\left(2\pi\frac{t}{T}\right)$$
where $0\leq x \leq L$

\subsubsection{Strings}

[Video demo of standing wave on a string.]

[Diagrams showing various standing waves and associated wavelengths.]\nopagebreak
\vspace{8cm}

Only certain wavelengths/frequencies are possible. For strings that are fixed on both ends, the possible solutions are
  $$\lambda_m=\frac{2L}{m}$$
  where $m=1,2,3,\dots$ are the harmonics. This corresponds to frequencies of
  $$f_m=m\left(\frac{v}{2L}\right)=mf_1$$
  where $v$ is the wave speed.

\subsubsection{Open-open tubes}
Standing waves can also form in tubes. These standing waves are longitudinal waves resulting from compression and rarefaction of air. Here, we describe the waves in terms of variations in pressure. Our constraint is that the pressure on the open end of tubes is (approximately) equal to atmospheric pressure. 

[Diagrams for open-open tube, with pressure fixed on both ends.]\nopagebreak
\vspace{8cm}

The same possible wavelengths are possible as for a string that is fixed on both ends, and the pressure could be described as

$$\Delta P(x,t)=A\sin\left(2\pi\frac{x}{\lambda}\right)\sin\left(2\pi\frac{t}{T}\right)$$
where $0\leq x \leq L$.

\subsubsection{Open-closed tubes}
The situation is a little different for open-closed tubes because the pressure on the closed end of the tube can differ from atmospheric pressure.

[Diagrams for open-open tube, with pressure fixed on end.]\nopagebreak
\vspace{8cm}


For open-closed tubes, we have a different set of harmonics than we had for open-open tubes:
$$\lambda_m=\frac{4L}{m}$$
where $m=1,3,5,\dots$. Note that open-closed tubes only have odd-numbered modes. This corresponds to frequencies of
  $$f_m=m\left(\frac{v}{4L}\right)=mf_1$$

It turns out that you can use the same wave model equation to describe the standing waves in this situation, but with a different relationship for $\lambda$.

\subsubsection{Wave speed}
Recall:
\begin{itemize}
\item For waves on strings,
  $$v=\sqrt\frac{F_t}{\mu}$$
  and for sound waves in air
  $$v=\sqrt\frac{\gamma RT}{M}$$
\end{itemize}
  
\subsection{Musical notes}
I'd like to finish our discussion of waves by talking a bit about musical notes.

Simples notes:
\begin{itemize}
\item note1.wav: $f_1=220\mbox{ Hz}$ (note "A")
\item note2.wav: $f_1=440\mbox{ Hz}$ (note "A", one octave higher)
\item note3.wav: $f_1=880\mbox{ Hz}$ (note "A", two octaves higher)
\end{itemize}

Beats:
\begin{itemize}
\item beats1.wav: $f_1=400\mbox{ Hz}$; $f_2=410\mbox{ Hz}$; $f_{\rm beat}=10\mbox{ Hz} \Rightarrow T=0.1\mbox{ s} \Rightarrow \mbox{ can't hear beats}$
\item beats2.wav: $f_1=400\mbox{ Hz}$; $f_2=401\mbox{ Hz}$; $f_{\rm beat}=1\mbox{ Hz} \Rightarrow T=1\mbox{ s}$
\item beats3.wav: $f_1=400\mbox{ Hz}$; $f_2=400.5\mbox{ Hz}$; $f_{\rm beat}=0.5\mbox{ Hz} \Rightarrow T=2\mbox{ s}$
\item beats4.wav: $f_1=100\mbox{ Hz}$; $f_2=101\mbox{ Hz}$; $f_{\rm beat}=1\mbox{ Hz} \Rightarrow T=2\mbox{ s}$
\end{itemize}

Harmonics:
\begin{itemize}
\item harmonic\_A.wav: $f_1=220\mbox{ Hz}$ (dropped an octave), $f_2=440\mbox{ Hz}$, $f_3=660\mbox{ Hz}$, $f_4=880\mbox{ Hz}$, $f_5=1100\mbox{ Hz}$
\item harmonic\_Csharp.wav: $f_1=277.2\mbox{ Hz}$, $f_2=554.4\mbox{ Hz}$, $f_3=831.6\mbox{ Hz}$, $f_4=1108.8\mbox{ Hz}$, $f_5=1386\mbox{ Hz}$
\item harmonic\_D.wav: $f_1=293.33\mbox{ Hz}$ (dropped an octave), $f_2=586.67\mbox{ Hz}$, $f_3=880\mbox{ Hz}$, $f_4=1173.33\mbox{ Hz}$, $f_5=1466.67\mbox{ Hz}$
\item harmonic\_E.wav: $f_1=330\mbox{ Hz}$, $f_2=660\mbox{ Hz}$, $f_3=990\mbox{ Hz}$, $f_4=1320\mbox{ Hz}$, $f_5=1650\mbox{ Hz}$
\end{itemize}

Consonance, dissonance, and chords
\begin{itemize}
\item consonance\_fifth\_AE.wav: combine A and E; some harmonics match and sounds good 
$$\left(f_1\right)_E=3/2*\left(f_1\right)_A$$
\item consonance\_fourth\_AD.wav: combine A and D; some harmonics match and sounds good 
  $$\left(f_1\right)_D=4/3*\left(f_1\right)_A$$
\item triad\_chord.wav: root, third, and fifth: A, C$^\#$, and E
\item dissonance\_AAsharp.wav: simultaneously play two adjacent notes (A and A$^\#$); feels unsettled or unresolved. Dissonance is not necessarily a bad thing; its basically beats, but the beat frequency is below hearing threshold (i.e., in the infrasound). Infrasound has been used in movies to produce an unsettled feeling
\end{itemize}
  
This has just been a brief intro to music theory. But, basically, the following things determine notes:
\begin{itemize}
\item natural frequencies of vibration
  \begin{itemize}
  \item tube length
  \item open-open vs open-closed
  \end{itemize}
\item number of overtones/harmonics
\item ``preciseness'' of overtones
\item timbre: fundamental frequency might not be most significant harmonic; how sounds decay with time;
\end{itemize}
