\section{Introduction to energy}
Objectives:
\begin{itemize}
\item Definitions
\item Work
\item Kinetic energy
\item Gravitational potential energy
\end{itemize}

\subsection{Work-energy theorem}
Who has heard of energy? What is energy? What kinds of energy have you heard of?

Don't worry, you're not the only one that doesn't understand energy. Here is a quote from Richard Feynman, a Nobel prize winner in physics and one of the more colorful characters in physics.

\begin{quote}
It is important to realize that in physics today, we have no knowledge of what energy is. We do not have a picture that energy comes in little blobs of a definite amount. It is not that way. However, there are formulas for calculating some numerical quantity\ldots It is an abstract thing in that it does not tell us the mechanism or the reasons for the various formulas.
\attrib{Richard Feynman} 
\end{quote}

Energy is
\begin{itemize}
\item an indirectly observed scalar (not vector!) quantity
\item the ability of one system to do work on another system
\end{itemize}

The concept of energy is extremely powerful because, like momentum, it allows us to ignore some complex interactions. Even more importantly, energy is what allows us to bridge together different fields of science. To me, this is where physics starts to become really exciting. Although energy is a rather abstract idea, it is also extremely important. 

There are many different types of energy:
\begin{itemize}
\item kinetic energy, $K$
\item potential (stored) energy, $U$ (we'll discuss a couple of types of potential energy)
\item chemical energy, $E_{chem}$
\item thermal energy, $E_{th}$
\item ...
\end{itemize}

We are going to define the total energy of a system as the sum of all of these energies,
$$E=K+U+E_{chem}+E_{th}+...$$

Energy can be transformed from one form to another (within the system). Can you think of some examples?
\begin{itemize}
\item The chemical energy stored in wood can be transformed into thermal energy.
\item The chemical energy of food is converted into kinetic energy when you move.
\item A falling object loses potential energy and gains kinetic energy.
\end{itemize}

Some transformations are ``easier'' than others. Its ``easy'' to create thermal energy, but difficult to turn thermal energy into potential energy or kinetic energy.

\textit{Energy transformation}: Changes in energy within a system.\\
\textit{Energy transfer}: Energy exchange between the system and the environment.\\
\textit{Environment}: Everything not in the system...

Just like when we talked about momentum, we can define the system in any way that is convenient.

Energy transfer occurs as ``work'' (i.e., a mechanical transfer of energy) or ``heat'' (i.e., a thermodynamic transfer of energy due to a temperature difference).

If there is no heat being transferred into the system, then the work done on a system changes the energy of the system
$$\Delta E=W$$
This means that
$$\Delta{E}=\Delta{K}+\Delta{U}+\Delta{E_{chem}}+\Delta{E_{th}}+...=W.$$
This is referred to as the work-energy theorem.

If our system is isolated (i.e., there is no work being done to or by the system), $W=0$ and 
$$\Delta{E}=0.$$
Any idea what this is referred to? Its the conservation of energy, one of the most powerful concepts in classical mechanics.

OK, so this is all nice and dandy, but what are $K$, $U$, $E_{chem}$, $E_{th}$, $W$, $...$?

\subsection{Work}
Let's first discuss work, the mechanical transfer of energy by an external force. We'll start by considering a constant external force, and we'll define work as 
$$W=Fd,$$
where $F$ is the external force applied to the system and $d$ is the displacement of the system. $F$ has to be parallel to $d$; if it isn't, we need to find the component of $F$ that is parallel to $d$. Work has units of joules, with $1\mbox{ J}=1\mbox{ N}\cdot\mbox{m}$. Don't confuse this with torque, even though the units are the same!

By the work-energy theorem, this means that
$$\Delta E=Fd.$$

\subsubsection{Example \#1: Pushing a book across a table}
You push your physics book across the table at a constant rate. How much work do you do to the book?

First, how much force are you exerting?
$$\sum F_x = F - F_k = ma = 0$$
$$F = F_k = \mu_kF_n$$

$$\sum F_y = F_n - F_g = 0$$
$$F_n = F_g = mg$$

and therefore
$$F = \mu_kmg$$

$$W=Fd=\mu_kmgd$$

If the book weighs 1~kg, $\mu_k=0.3$, and $d=1$~m, then $\boxed{W=2.9\mbox{ J}}$.

You then drop the book into the garbage. How much work does gravity do to the book?

$$W_g=-F_g\Delta{y}=-mg\Delta{y}$$

We need to be careful here. From the definition of work, we need the force that points in the direction of displacement. If our coordinate system points up, then $F_g$ points in the negative direction and $\Delta{y}<0$. If the book weighs 1 kg and it falls a distance of 1 m, the work done by gravity is 
$$\boxed{W_g=9.81\mbox{ J}}.$$

\subsubsection{Example \#2: Pulling a crate}
You pull a crate with a rope at a constant velocity for 3 m. The tension in the rope is 70 N. How much work is done to the crate by the rope, and where does this energy go?

[Insert diagram.]
\vspace{5cm}

$$W=F_t\cos\theta\cdot{d}=182\mbox{ J}$$

As we'll see, the kinetic energy and potential energy of the crate is unchanged; the 182 J goes into thermal energy from friction.


\subsection{Kinetic Energy}
Let's consider what happens if you move an object with constant force and no friction or drag. For example, we could be using a rope to pull a box across a frictionless surface.
$$\sum F_x=F_{ext}=ma_x\Rightarrow a_x$$
The work that is done by force $F_{ext}$ is
$$W = F_{ext}\Delta x = ma\Delta x$$

Recall from kinematics that
$$2a\Delta x = v_f\ds^2-v_i\ds^2$$
Plugging this in and rearranging, we find that

$$W=\frac{1}{2}mv_f\ds^2-\frac{1}{2}mv_i\ds^2$$

Let's define the kinetic energy (energy of motion) as
$$\boxed{K=\frac{1}{2}mv^2}$$
Note that $v$ is the \textit{speed}, not the velocity vector. 

In this particular example,
$$W=\Delta{K}.$$
We can do a similar analysis for angular motion, and we find that
$$\boxed{K_{rot}=\frac{1}{2}I\omega^2}$$
The total kinetic energy of an object is
$$K_{total}=K_{trans}+K_{rot}.$$

\subsection{Gravitational Potential Energy}
What if instead we had moved the object vertically some distance $\Delta y$ at a constant speed with some force $F_{ext}$?
$$\sum F_y=F_{ext}-F_g=ma_y$$
Since the object is moving at a constant speed, $a_y=0$ and therefore
$$F_{ext}=F_g=mg.$$
The work done on the object is
$$W=F_{ext}\Delta{y}=mg\Delta{y}=\Delta{E}.$$
We will call
$$\boxed{\Delta{U_g}=mg\Delta{y}}$$
the change in gravitational potential energy (or often, just potential energy). Potential energy is always relative to some reference state, and so we always have to talk about changes in potential energy. Notice also that the path that we take doesn't affect the solution.

Potential energy is a stored energy; forces that store energy are called conservative.

What if we hadn't required that $a_y=0$?
$$\sum F_y=F_{ext}-F_g=ma_y$$
$$F_{ext}=mg+ma_y$$
$$W = F_{ext}\Delta y = mg\Delta y + ma_y\Delta y$$
Using results from above,
$$W = mg\Delta y + \frac{1}{2}m\left(v_f\ds^2-v_i\ds^2\right)$$
$$W = \Delta U_g + \Delta K$$


\subsubsection{Example \#3: Ball dropped from rest}
A ball is dropped from a height of 1 m. What is its speed right before it hits the ground?

The old method, using forces and kinematics:
$$\sum F_y=-F_g=ma_y\Rightarrow a_y=-g$$
$$v_f\ds^2-v_i\ds^2=2a_y\Delta{y}\Rightarrow v_f\ds^2=-2g\Delta{y}\Rightarrow \boxed{v_f=\sqrt{-2g\Delta{y}}}$$

The new method, using conservation of energy (define system to include ball and the Earth):
$$\Delta{E}=\Delta{K}+\Delta{U_g}=0$$
$$\frac{1}{2}mv_f\ds^2-\frac{1}{2}mv_i\ds^2+mg\Delta{y}=0\Rightarrow \boxed{v_f=\sqrt{-2g\Delta{y}}}$$

\subsubsection{Example \#4: Box sliding down a frictionless ramp}
A box slides down a frictionless ramp. What is its speed at the end of the ramp?

The old method, using forces and kinematics:
$$\sum F_x=F_g\sin\theta=ma_x\Rightarrow a_y=-g\sin\theta$$
Then, same as with the previous example,
$$v_f=\sqrt{-2g\sin\theta\Delta{y}},$$
but $\Delta{y}=-H/\sin\theta$, so
$$\boxed{v_f=\sqrt{-2gH}}$$

The new method, using conservation of energy (define system to include the ramp and the box):
$$\Delta K+\Delta U_g=0=\frac{1}{2}mv_f\ds^2-\frac{1}{2}mv_i\ds^2+mg\Delta{y}$$
$H=-\Delta{y}$ and $v_i=0$, so
$$\boxed{v_f=\sqrt{-2gH}}$$

\clearpage
