\section{Motion in two dimensions}
Objectives:
\begin{itemize}
\item Vectors
\item Projectile motion
\end{itemize}

Demonstrations:
\begin{itemize}
\item Free-fall apparatus: one marble falls straight down while the other is shot horizontally
\end{itemize}

\subsection{Kinematic equations in 1D}
In the first two lectures I introduced the kinematic equations and variables used to describe motion in 1-dimension for the case of constant acceleration. Recall:
$$\Delta{v}=a\Delta{t}$$
$$\Delta{x}=v_i\Delta{t}+\frac{1}{2}a\Delta{t}\ds^2$$
$$v_f\ds^2-v_i\ds^2=2a\Delta{x}$$
We saw that one special case of constant acceleration is that of gravity near the Earth's surface, for which $g=9.81\mbox{ m/s}^2$. $g$ is always positive and always points downward. Acceleration may be $\pm g$, depending on the orientation of the coordinate system.

We also saw that if $a=0$, this reduces to the kinematic equation for constant velocity, and so
$$\Delta{x}=v\Delta{t}$$
We have been using $x$ to define our coordinate system. The coordinate system can point in any convenient direction, and sometimes we will use $y$ or $z$ to indicate a distance along an axis.

\subsection{Vectors}
We will often want to describe motion in 2-dimensions, and sometimes in 3-dimensions. In these instances we will need to use \textit{vectors}, and we will need to make use of $y$ and/or $z$. It is pretty straightforward to generalize what we have already learned to describe motion in 2- and 3-dimensions. Before continuing, we also need to remember what is meant by a vector.

Vector: a geometric quantity having both magnitude \textit{and} direction.

[Insert diagram of a vector.]

\clearpage
\begin{table}[h]
\begin{tabular}{lll}
\textbf{Kinematic variable} & \textbf{1D (scalar quantity)} & \textbf{2D (vector quantity)}\\
\hline
position & $x$ & $\vec{x}=\langle{x,y}\rangle$\\
displacement & $\Delta{x}$ & $\Delta\vec{x}=\langle{\Delta{x},\Delta{y}}\rangle$\\
velocity & $v=\frac{\Delta{x}}{\Delta{t}}$ & $\vec{v}=\langle{v_x,v_y}\rangle=\langle{\frac{\Delta{x}}{\Delta{t}},\frac{\Delta{y}}{\Delta{t}}}\rangle$\\
acceleration & $a=\frac{\Delta{v}}{\Delta{t}}$ & $\vec{a}=\langle{a_x,a_y}\rangle=\langle{\frac{\Delta{v_x}}{\Delta{t}},\frac{\Delta{v_y}}{\Delta{t}}}\rangle$\\
\hline
\end{tabular}
\end{table}

We can do this as long as the $x$- and $y$-axes are orthogonal/perpendicular to each other. For example, if an object moves in the $x$-direction, its $y$-position doesn't necessarily change at the same time. This doesn't mean that motion in the $x$-direction is independent of motion in the $y$-direction. 

[Insert diagram showing displacement vector. Should have $\vec{x_1}$ and $\vec{x_2}$]
\vspace{6cm}


Initial position: $\vec{x}_i=\langle{x_i,y_i}\rangle$\\
Magnitude: $\vec{x}_i=|\vec{x}_i|=\sqrt{x_i\ds^2+y_i\ds^2}$\\
Angle: $\tan\theta_i=y_i/x_i$\\
If angle is known: $x_i=\cos\theta_i$ and $y_i=\sin\theta_i$\\

After some time, the object moves to position $\vec{x}_f$. The displacement is $$\Delta\vec{x}=\vec{x}_f-\vec{x}_i=\langle{x_f,y_f}\rangle-\langle{x_i,y_i}\rangle=\langle{x_f-x_i,y_f-y_i}\rangle.$$ 
\textbf{You have to add or subtract vector components!} If this is confusing, ask questions and read your textbook. We will use vectors throughout the semester.

You can also add vectors graphically. Slide the tail of one vector to the tip of the other vector. For subtraction, the easiest way is to recall that subtraction is the same as adding a negative number. This applies also for vectors.

\clearpage
[Insert diagrams showing addition and subtraction of vectors.]
\vspace{6cm}

Anyway, back to representing motion in 2-dimensions. In the majority of the problems that we will consider we will be able to treat motion in the $x$- and $y$- directions as being independent. In this case, we can directly use our kinematic equations to describe 2-dimensional motion.

\begin{table}[h]
\begin{tabular}{ll}
\textbf{Motion in $x$-direction}\hspace{3cm} & \textbf{Motion in $y$-direction}\\
\hline
$\Delta{v_x}=a_x\Delta{t}$ & $\Delta{v_y}=a_y\Delta{t}$\\
$\Delta{x}=v_{x,i}\Delta{t}+\frac{1}{2}a_x\Delta{t}\ds^2$ & $\Delta{y}=v_{y,i}\Delta{t}+\frac{1}{2}a_y\Delta{t}\ds^2$\\
$v_{x,f}\ds^2-v_{x,i}\ds^2=2a_x\Delta{x}$ & $v_{y,f}\ds^2-v_{y,i}\ds^2=2a_y\Delta{y}$\\
\hline
\end{tabular}
\end{table}

We really haven't added all that much complexity here. All we're saying is that the equations that we developed for 1-dimensional motion can be generalized to describe motion in 2-dimensions.


\subsection{Demo: Free-fall apparatus}
Which marble hits the ground first?

Last class we started talking about projectile motion, which involves motion of objects in two-dimensions. I'd like to continue that discussion with a demo, after which we'll discuss other types of two-dimensional motion.

Which marble will hit the ground first? The one that falls straight down, or the one that is kicked by the spring?

Turns out that they hit the ground at the same time! They both start with no vertical velocity, and gravity acts downward on both marbles equally.

If we measure the height that the marbles are dropped from, and the distance that the projectile marble travels, can we calculate the marble's initial speed?

First, let's consider motion in the $y-$direction to determine the time it takes the marble to hit the ground.
$$\Delta{y}=v_{y,i}\Delta{t}+\frac{1}{2}a\Delta{t}\ds^2$$
But $v_{y,i}=0$, $\Delta{y}=-H$, and, if the $y$ points upward, $a=-g$.
$$-H=-\frac{1}{2}g\Delta{t}\ds^2$$
$$\Delta{t}\ds^2=\frac{2H}{g}$$
$$\Delta{t}=\sqrt{\frac{2H}{g}}$$

Now let's use the horizontal distance that the marble travelled to calculate its initial velocity. There is no acceleration in the $x-$direction.
$$v=\frac{\Delta{x}}{\Delta{t}}=\frac{\Delta{x}}{\sqrt{\frac{2H}{g}}}=\Delta{x}\sqrt{\frac{g}{2H}}$$

\subsection{Example: What angle to throw a ball?}
You throw a ball on a level field. At what angle from horizontal should you throw the ball to get the maximum distance out of the throw?

Let's call $V$ the initial \textit{speed}. It is a positive value and doesn't indicate the direction of motion. We'll define $x$ to be the horizontal direction and $y$ to be the vertical direction. So this gives

$$\vec{v}_i=\langle{v_x,v_y}\rangle=\langle{V\cos\theta_i,V\sin\theta_i}\rangle$$

First we need to figure out how long it takes the ball to hit the ground. We only need to worry about the $y-$direction, because the motion in the $x-$direction doesn't affect how long it takes the ball to hit the ground (as long as we don't have to worry about lift).

Let's call $\Delta{t}$ the time to hit the ground, in which case $\Delta{y}=0$.
$$v_{y,f}\ds^2-v_{y,i}\ds^2=2a_y\Delta{y}\Rightarrow v_{y,f}=-v_{y,i}$$
Interesting!
$$\Delta{v_y}=a_y\Delta{t}$$
$$v_{y,f}-v_{y,i}=-2v_{y,i}=-2V\sin\theta_i=-g\Delta{t}$$
$$\Delta{t}=\frac{2V\sin\theta_i}{g}$$

Now, how far does the ball travel in the $x-$direction during this amount of time? The velocity in the $x-$direction is constant, so
$$\Delta{x}=v_x\Delta{t}=\frac{2V^2\cos\theta_i\sin\theta_i}{g}$$

For what $\theta_i$ is $\Delta{x}$ a maximum? Make a plot.

[Insert plot of $\Delta{x}$.]
\vspace{4cm}


Maximum occurs when $\theta_i=45^\circ$.

(This has assumed that air resistance is negligible, and so $v_x$ is constant. If we had accounted for air resistance, which is a \textit{really} difficult problem, we would have found that $\max{\theta_i}<45^\circ$.)

\clearpage
