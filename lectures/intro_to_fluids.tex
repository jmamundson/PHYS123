\section{Introduction to fluids}
Objectives:
\begin{itemize}
\item Hydrostatics
  \begin{itemize}
  \item Density
  \item Pressure
  \end{itemize}
\end{itemize}

\subsection{Review of thermodynamics of gases}
Today I'd like to start talking about fluids, but before doing that I think we should do an example related to the last lecture.

Important ideas from last time:
\begin{itemize}
\item Ideal gas law: $PV=nRT$. For closed containers, we can saw that $\frac{PV}{T}=\mbox{constant}$.
\item Pressure at sea level: $P=101.325$~kPa; this is equivalent to 101.325~kN acting on 1~m$^2$ or 10.1325~N on 1~cm$^2$
\item Change in thermal energy of a gas is given by $\Delta{E_{th}}=\frac{3}{2}nR\Delta{T}$
\item Work and heat (transfer) can change the total energy of a gas
\item Isochoric or isovolumetric process: heat transferred into a gas container with no change in volume
\item Isobaric process: pressure is constant, but volume and temperature can change due to work or heat transfer
\item Isothermal process: no change in temperature (or thermal energy), so work done by a gas must balance with heat transfer into the gas
\item The specific heat of a gas depends on whether the volume or pressure change...
\end{itemize}

\subsubsection{Example \#1: expandable cube}
An expandable cube, initially 20 cm on each side, contains 3.0 g of helium at 20$^\circ$C. 1000 J of heat is transferred into this gas. What are (a) the final pressure if the process is at a constant volume and (b) the final volume if the process is at constant pressure?

Also given:\\
3 g of He is 0.75 mol\\
$C_v=12.5$ J/(mol$\cdot$K)\\
$C_p=20.8$ J/(mol$\cdot$K)

(a) Because this is at constant volume, and because
$$\frac{PV}{T}=\mbox{constant},$$
we have
$$\frac{P}{T}=\mbox{constant}\Rightarrow\frac{P_i}{T_i}=\frac{P_f}{T_f}.$$
We want to solve for the final pressure, so
$$P_f=\frac{T_f}{T_i}P_i.$$

We know that $T_i=293\mbox{ K}$. What are $T_f$ and $P_i$? Heating the gas will change its temperature
$$Q=nC_v\Delta{T}\Rightarrow \Delta{T}=\frac{Q}{nC_v}=107\mbox{ K},$$
so $T_f=400$ K. The initial pressure we can find using the ideal gas law.
$$P_iV_i=nRT_i\Rightarrow P_i=\frac{nRT_i}{V_i}=228\mbox{ kPa}$$
Plugging this into the equation for $P_f$ gives
$$\boxed{P_f=311\mbox{ kPa}}$$

(b) Because this is constant pressure, we have
$$\frac{PV}{T}=\mbox{constant}\Rightarrow \frac{V}{T}=\mbox{constant}$$
This means that
$$\frac{V_i}{T_i}=\frac{V_f}{T_f}\Rightarrow V_f=\frac{T_f}{T_i}V_i$$

The initial volume and temperature are given. We just need to calculate $T_f$.
$$Q=nC_p\Delta{T}\Rightarrow \Delta{T}=\frac{Q}{nC_p}=64.1\mbox{ K},$$
so $T_f=357\mbox{ K}$. Consequently, 
$$\boxed{V_f=0.00975\mbox{ m}^3=9.75\mbox{ L}}$$.

\subsection{Hydrostatics}
That's all that I have to say (for now, anyway) about thermodynamics and gases. I'd like to now turn our attention to fluids. We'll first focus on hydrostatics --- fluids at rest.

What is a fluid? Its a substance that flows and takes the shape of a container.

[Diagram of container containing a fluid.]\nopagebreak
\vspace{5cm}

In a fluid, molecules are weakly bonded but can slide past each other.

\subsection{Density}
An important parameter for describing fluids is density.
$$\rho=\frac{m}{V}$$

Sometimes $\rho$ depends on pressure, sometimes it doesn't...

Examples:\\
sea water: $\sim 1030$ kg/m$^3$; it depends on salinity, temperature, and pressure

fresh water: $\sim 1000$ kg/m$^3$; it depends on temperature\\
\begin{itemize}
\item at 20$^\circ$C, $\rho=998.2071$ kg/m$^3$
\item at 4$^\circ$C, $\rho=999.9720$ kg/m$^3$
\item at 0$^\circ$C, $\rho=998.8395$ kg/m$^3$
\end{itemize} 

ice: $\sim 917$ kg/m$^3$ (if bubble free, otherwise $<917$ kg/m$^3$)

firn: $\sim 800$ kg/m$^3$

fresh snow: $100$--$300$ kg/m$^3$

Water is \underline{weird}. For most substances, the solid form is more dense than the liquid form. If this were also true for water, life wouldn't exist --- at least not as we know it. Its also weird that the densest fresh water is 4$^\circ$C.

Ice that forms in a water body floats at the surface, and it insulates the water from cold air. The water at depth remains liquid.

[Draw diagram.]\nopagebreak
\vspace{5cm}

Its very difficult to form thick lake ice or sea ice. ``Multi-year'' ice in the Arctic Ocean is just a few meters thick.

One important process involving water is the overturning of lakes during spring and fall, which is due to the fact that the densest water is at 4$^\circ$C.

[Draw diagrams showing overturning lakes and temperature profiles.]\nopagebreak
\vspace{8cm}

\subsection{Hydrostatic pressure}
Why does dense material sink? To answer that, we need to think a bit about pressure.

[Insert diagram with a column of water in a container.]\nopagebreak
\vspace{5cm}

Let's assume that the column is in hydrostatic equilibrium.

There is a force from the overlying water and air that acts on downward on the column of water. This force is equal to $$F_{down}=P_0A + mg=\rho Vg=\rho hAg$$

From Newton's Third Law, there is a force pushing back on the column with an equal and opposite force. If, as before, we define pressure as $P=F/A$, then the force upward force exerted by the parcel of water is
$$F=PA=P_0A+\rho hAg$$
Dividing by $A$ gives
$$\boxed{P=P_0+\rho gh}$$
The hydrostatic fluid pressure depends on depth.

This pressure acts equally in all directions. If that wasn't the case, the fluid parcel would change shape. (One way to check would be to put a pressure sensor in water -- basically a device with a spring that responds to the force from the water -- and rotate it about in the water column.) This tells us that in a static fluid, water pressure is constant along horizontal lines and the water surface rises to the same level everywhere.

I always think its nice to have a reference value for physical units. A convenient one for pressure is to think about the water pressure below 1 m of water.
$$P=\rho gh=1000\mbox{ kg/m}^3\times 9.81\mbox{ m/s}^2\times 1\mbox{ m}\approx 10^4\mbox{ Pa}$$

Note that the water pressure below 10 m of water is 10$^5$ Pa, which is basically the same as the atmospheric pressure at sea level. In other words, 10 m of water is equivalent to several kilometers of air.

Gas pressure also varies with elevation for the same reason as fluid pressure. However, since the density of gases are so low, the pressure variations are small over short distances. In a closed container, we treat gas pressure as constant.

\subsubsection{Pressure demos}
\begin{itemize}
\item Bottle with pin holes at different elevations to show that pressure varies with depth
\item Pipette to illustrate water pressure in a vacuum
  $$P_0=\rho g d + P_{bubble}$$
  $$P_{bubble} = P_0-\rho g d$$
  If $d=10$~cm and $\rho=1000$~kg/m$^3$, then $P_{bubble}=100.344$~kPa
\end{itemize}


\clearpage
