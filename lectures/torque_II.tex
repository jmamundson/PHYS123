\section{Torque II}
Objectives:
\begin{itemize}
\item Pulleys
\item Rolling motion
\end{itemize}

\subsection{Background}
Last class I introduced the idea of torque.
$$\tau=rF_\perp$$
Forces generate torques, and the magnitude of the torque depends on (1) the distance from the axis of rotation to the point at which the force is acting and (2) the component of the force that is perpendicular to the lever arm. Torques are positive for counter-clockwise rotation.

We also saw that torques cause angular acceleration, such that
$$\sum \tau=I\alpha$$
where $I$ is the moment of inertia. $I$ can be thought of as an object's resistance to angular acceleration. This equation is Newton's Second Law for rotation.

Until now we've (implicitly) ignored objects' moments of inertia. Today I'd like to go through a couple of more involved problems that involve the moment of inertia.

\subsection{Demo of coins on a falling meter stick}
Elements of this problem:
\begin{itemize}
\item Gravitational torque: $-\frac{1}{2}Lmg = I\alpha$
\item Moment of inertia: $I = \frac{1}{3}mL^2$
\item $\alpha$ is the same everywhere, but $a_y$ isn't!
\end{itemize}



\subsection{Example \#1: Block pulled by a string}
A block is pulled by a string across a horizontal, frictionless plane. The other end of the string is suspended over a pulley and attached to a hanging mass. We will not assume that the pulley is massless but we will assume that the pulley is frictionless. Because the pulley has a moment of inertia, the tension in the two parts of the string are not necessarily equal. What is the acceleration of the block?

\clearpage
[Insert diagram.]
\vspace{8cm}


Start by summing the forces acting on mass $m_1$.
\begin{equation}\sum F_x=T_1=m_1a_x\end{equation}
We want to find $a_x$. 
This is one equation with two unknowns. We need at least one more equation.

Now let's sum the forces acting on mass $m_2$.
$$\sum F_y=T_2-F_{g,2}=m_2a_y$$
Note that $a_x$ and $a_y$ have the same magnitude; $a_x>0$ and $a_y<0$, so $a_y=-a_x$. Therefore,
\begin{equation}T_2-m_2g=-m_2a_x\end{equation}
We've introduced another equation, but also one more unknown. We now have two equations and three unknowns.

We can come up with a third equation by summing the torques on the pulley.
$$\sum \tau=rT_1-rT_2=I\alpha\Rightarrow T_2-T_1=I\frac{\alpha}{r}$$
This equation has given us yet one more unknown, $\alpha$. But we saw earlier that $a_t=\alpha r$, so $\alpha=a_t/r$. We need to be careful with the signs here. We have defined our torques to be positive for counter-clockwise rotation. We have also defined $a_x$ to be positive if it causes clockwise rotation. Therefore, we want to set $\alpha=-a_x/r$. Thus,
\begin{equation}T_1-T_2=-I\frac{a_x}{r^2}\end{equation}
We have three equations and three unknowns. Let's solve the first two equations for $T_1$ and $T_2$, respectively, and insert them into the third equation.

$$T_1=m_1a_x$$
$$T_2=m_2g-m_2a_x$$

And so
$$m_1a_x-m_2g+m_2a_x=-I\frac{a_x}{r^2}$$
We want to solve this for $a_x$.
$$-m_2g=-m_2a_x-m_1a_x-I\frac{a_x}{r^2}=-\left(m_2+m_1+\frac{I}{r^2}\right)a_x$$
Finally, we arrive at
$${a_x=\frac{m_2g}{m_2+m_1+\frac{I}{r^2}}}$$

The moment of inertia of a disk is $I=(mr^2)/2$. If we treat the pulley as being a uniform disk of mass $m_p$, we can simplify the above equation to 
\begin{equation}\boxed{a_x=\frac{m_2g}{m_2+m_1+\frac{m_p}{2}}}\end{equation}

Aside:

If you look back in your notes, you'll see that when we ignored the mass of the pulley, we arrived at
$$a_x=\frac{m_2g}{m_2+m_1}$$
Furthermore, if we plug this back into the original force balance equations, we see that 
$$T_1=\frac{m_1m_2g}{m_2+m_1}$$
and
$$T_2=m_2g-\frac{m_2^2g}{m_2+m_1}=\frac{m_2^2g+m_1m_2g}{m_2+m_1}-\frac{m_2^2g}{m_2+m_1}=\frac{m_1m_2g}{m_2+m_1}$$
In other words, if $m_p=0$ then $T_1=T_2$.

How much error do we introduce by assuming that $m_p=0$? Let's plug in some numbers to find out.

$m_1=0.2\mbox{ kg}$\\
$m_2=0.1\mbox{ kg}$\\
$m_p=0.01\mbox{ kg}$

$a_x\mbox{(with pulley's mass)}=3.22\mbox{ m/s}^2$\\
$a_x\mbox{(without pulley's mass)}=3.27\mbox{ m/s}^2$

That's an error of less than 2$\%$.

\subsection{Example \#2: Ball rolling down a hill}
Let's consider a ball rolling down a ramp without slipping. First, recall the rolling constraint. Note also that rolling motion can be thought of as a combination of linear translational motion and rotational motion around the object's center of mass.

[Insert diagram of rolling object.]
\vspace{5cm}

[Insert diagram.]
\vspace{5cm}

\begin{equation}\sum F_x=F_g\sin\theta-F_f=ma \end{equation}

How do we deal with the frictional force? The ball isn't slipping, yet it is still moving... One way to do this is to consider a reference frame that is moving with the ball, and then calculate the torques around its center.

\begin{equation}\sum\tau=-rF_f=I\alpha \end{equation}
We have seen that $\alpha=a_t/r$. In this case, $a>0$ implies $\alpha<0$, so $\alpha=-a/r$ and therefore
$$-rF_f=-I\frac{a}{r}\Rightarrow F_f=I\frac{a}{r^2}$$
Inserting this into the above equation gives
$$F_g\sin\theta-I\frac{a}{r^2}=ma\Rightarrow a\left(\frac{I}{r^2}+m\right)=F_g\sin\theta=mg\sin\theta$$
$$a=\frac{mg\sin\theta}{\frac{I}{r^2}+m}$$
For a solid, uniform sphere, $I=(2/5)mr^2$. Thus,
$$a=\frac{mg\sin\theta}{\frac{(2/5)mr^2}{r^2}+m}=\frac{mg\sin\theta}{(2/5)m+m}=\frac{mg\sin\theta}{(7/5)m}$$
$$\boxed{a=\frac{5}{7}g\sin\theta}$$


\subsubsection*{Alternative approach}
The other way to solve this problem, if you're not comfortable thinking about the rotation of an object in a moving reference frame, is to sum the torques around the point where the ball is in contact with the ramp. Essentially, the gravitational force is not directly above this point and so the ball is constantly ``falling''. In this approach, the torque is entirely due to the component of the gravitational force that points down the ramp.

$$\sum\tau = -rF_g\sin\theta = I\alpha = -I\frac{a}{r}$$

The moment of inertia of a ball around its center of mass is $(2/5)mr^2$. Because the ball is not rotating around its center of mass in this reference frame, we need to use the parallel-axis theorem, so that
$$I = \frac{2}{5}mr^2 + mr^2 = \frac{7}{5}mr^2$$

Plugging this in, we find that
$$-rmg\sin\theta = -\frac{7}{5}mr^2\frac{a}{r} \Rightarrow \boxed{a = \frac{5}{7}g\sin\theta}$$

\subsubsection*{Kinematics}
If the ball starts at rest, how long will it take the ball to reach the bottom of the ramp and what will its speed be?

Recall:\\
$$v_f^2-v_i^2=2a\Delta{x}$$
$$v_f^2=2\frac{5}{7}g\sin\theta\frac{H}{\sin\theta}=\frac{10}{7}gH\Rightarrow \boxed{v_f=\sqrt{\frac{10}{7}gH}}$$


$$\Delta x=v_i\Delta{t}+\frac{1}{2}a\Delta{t}^2=\frac{1}{2}a\Delta{t}\ds^2$$
$$\frac{H}{\sin\theta}=\frac{1}{2}\frac{5}{7}g\sin\theta\Delta{t}\ds^2$$
$$\Delta{t}\ds^2=\frac{5}{14}\frac{g}{H}\sin\ds^2\theta$$

$$\boxed{\Delta{t}=\sqrt\frac{5g}{14H}\sin\theta}$$


\clearpage
