\section{Fluid dynamics}
Objectives:
\begin{itemize}
\item Ideal fluids
\item Flux continuity equation
\item Bernoulli's equation
\end{itemize}

\subsection{Pressure}
\begin{itemize}
\item Fluids: material that flows
\begin{itemize}
  \item Gases: highly compressible fluids
  \item Liquids: weakly compressible fluids with well-defined surfaces
\end{itemize}
\item Pressure = force / area; depends on weight of overlying fluid
  \begin{itemize}
  \item For liquids in hydrostatic equilibrium: $P=P_{atm}+\rho gh$ in an open container, or $P=P_{gas}+\rho gh$ in a closed container; assume that density is constant
  \item For gases (especially in closed containers): often assume constant pressure. Because the density is low, changes in pressure are small over short distances. If considering large distances, have to account for changes in density.  
  \item Pressure is constant along horizontal lines for a fluid that is in hydrostatic equilibrium.
  \end{itemize}
\end{itemize}

\subsection{Ideal fluids}
We will assume three things:
\begin{enumerate}
\item The fluid is incompressible. This is a good approximation for many liquids, especially water, and is not even a bad approximation for gases.
\item The fluid flow is steady (laminar) and non-turbulent.
  [Insert diagram.]
  \vspace{3cm}
  
The transition from laminar to turbulent flow depends on the fluid properties, the flow speed, and the geometry of the flow. [Discuss water coming from faucet.]
\item The fluid is non-viscous. Viscosity is a fluid's resistance to flow; think of it as internal friction. Water has a low viscosity, syrup has a high viscosity. (Later we'll take into account viscosity.)
\end{enumerate}

\subsection{Flux continuity}
These assumptions lead to a nice continuity equation for fluid flow through a \textit{pipe}:
$$Q=vA=\mbox{constant}$$
where $Q$ is the flux [m$^3$/s], $v$ is the velocity, and $A$ is the cross-sectional area. Note that for ideal fluids, the flow velocity does not vary across the channel.

[Insert diagram of a pipe showing that $Q_1=Q_2$.]\nopagebreak
\vspace{4cm}

The continuity equation is different for open channel flow, because the depth of the channel can also change. In that case,
$$\frac{\Delta V}{\Delta t}=Q_{in}-Q_{out}$$

\subsubsection{Example \#1: Water flow through variable diameter pipe}
Water enters a pipe with a diameter of 1 cm at 4 m/s. The pipe expands to 2 cm, then shrinks to 0.5 cm. (a) What is the flux? (b) What are the water speeds in each section of pipe?

(a) Flux:
$$Q=Av=\pi r^2v=\pi(0.05\mbox{ m})^2\times 4\mbox{ m/s}=0.03\mbox{ m}^3/s$$

(b) Speed at points 2 and 3:
$$v_2=\frac{A_1v_1}{A_2}=1\mbox{ m/s}$$
$$v_3=\frac{A_1v_1}{A_3}=16\mbox{ m/s}$$


\subsection{Bernoulli's Equation}
Ok, that's great, but what causes fluid motion? Fluid flow is constant through straight stretches of pipe, but accelerates or decelerates as the pipe diameter varies.

For an ideal fluid in a pipe, there are no external forces acting on the fluid --- only pressure.

Consider a small section of fluid in a pipe:

[Insert diagram.]\nopagebreak
\vspace{5cm}


$$\sum F_x=P_lA-P_rA=A(P_l-P_r)=A\Delta{P}=ma_x$$
$\Delta{P}$ represents the pressure difference along the pipe. If $\Delta{P}>0$, the water accelerates to the right. The larger the pressure difference, the greater the acceleration. This means that pressure is high where the fluid is moving slowly, and low where the fluid is moving quickly. As a result, pressure is low in narrow sections of pipe. This is referred to as the Bernoulli effect.

[Demo of Bernoulli effect.]

OK, so pressure gradients (differences) are one thing that causes fluid to flow. Gravity also causes fluids to flow. Pressure gradients and gravity both represent forces; we want to relate these things to fluid flow in a single equation. To do this, we'll use the work-energy theorem.

Recall:
$$W=\Delta E=\Delta K+\Delta U_g$$
Here we are assuming that the fluid is inviscid, so there is no change in thermal energy.

We're going to look at what happens to a volume of water as it flows through a pipe that changes elevation and has a change in area.

[Insert diagram.]\nopagebreak
\vspace{5cm}

As the slug of water moves through the pipe, there a loss of volume on the left and a gain in volume on the right. 
$$\Delta V_1=A_1\Delta{x_1}$$
or
$$\Delta V_2=A_2\Delta{x_2}$$
Due to incompressibility, $\Delta V_1=\Delta V_2=\Delta V$.

The change in kinetic energy is
$$\boxed{\Delta K=\frac{1}{2}\rho \Delta V(v_2\ds^2-v_1\ds^2)}$$

The change in potential energy is
$$\boxed{\Delta U_g=mg\Delta h=\rho \Delta V(h_2-h_1)}$$

Ok, so we have $\Delta{K}$ and $\Delta{U_g}$. How much work was done to the volume of water? Work done by pressure on the left was
$$W_1=F_1\Delta x_1=P_1A_1\Delta x_1=P_1\Delta V$$

Work done by pressure on the right is the opposite, so
$$W_2=-F_2\Delta x_2=-P_2A_2\Delta x_2=-P_2\Delta V$$

The net work is 
$$\boxed{W=W_1+W_2=(P_1-P_2)\Delta V}$$

Putting this all together,
$$(P_1-P_2)\Delta V=\frac{1}{2}\rho \Delta V(v_2\ds^2-v_1\ds^2)+\rho \Delta V(h_2-h_1)$$
Dividing by $\Delta V$ and re-arranging,
$$P_1+\frac{1}{2}\rho v_1\ds^2+\rho gh_1=P_2+\frac{1}{2}\rho v_2\ds^2+\rho gh_2=\mbox{constant}$$
This is referred to as Bernoulli's Equation.

This quantity is constant along a streamline (a path that a particle takes) in an ideal fluid.

Let's think about what this means by looking at the fluid flow through a pipe that has changes in elevation and diameter.

[Insert diagram.]\nopagebreak
\vspace{8cm}

\subsubsection{Example \#2: Water flow from a reservoir}
Water flows from a reservoir, through an intake tube, and down to a turbine. The intake tube has a diameter of 100 cm and is 50 m below the reservoir surface. The water drops 200 m to the turbine; water flows into the turbine through a 50 cm diameter nozzle. (a) What is the water speed into the turbine? (b) By how much does the inlet pressure differ from hydrostatic pressure?

(a) There is no flow at the surface of the reservoir, and the elevation at $y_3=0$. 
$$P_{\rm atm}+\rho g y_1=P_{\rm atm}+\frac{1}{2}\rho_g v_3^2$$
$$v_3=\sqrt{2gy_1}=70\mbox{ m/s}$$
Note that the speed decreases as the reservoir drains.

(b) The inlet pressure differs from hydrostatic pressure because the water is flowing. 
$$v_2A_2=v_3A_3$$
$$v_2\pi r_2^2=v_3\pi r_3^2$$
$$v_2=v_3\frac{r_3^2}{r_2^2}=v_3\left(\frac{r_3}{r_2}\right)^2=\frac{v_3}{4}=\frac{\sqrt{2gy_1}}{4}$$

Now we use Bernoulli's equation:
$$P_{\rm atm}+\rho g y_1=P_2+\frac{1}{2}\rho v_2^2+\rho g y_2$$
$$P_2=P_{\rm atm}+\rho g(y_1-y_2)-\frac{1}{2}\rho v_2^2$$
Notice that
$$P_{\rm static}=P_{\rm atm}+\rho g(y_1-y_2)$$
so the difference in pressure from hydrostatic pressure is simply
$$\frac{1}{2}\rho v_2^2=\frac{1}{2}\rho\frac{2gy_1}{16}=\frac{\rho gy_1}{16}=153000\mbox{ Pa}\approx 1.5\mbox{ atm}$$

The hydrostatic pressure at this depth is 5918000 Pa, or about 6 atm.



\clearpage
