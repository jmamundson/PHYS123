\section{Damped and driven oscillations}
Objectives:
\begin{itemize}
  \item Simple harmonic motion
  \item Damped oscillations
  \item Driven oscillations
\end{itemize}

\subsection{Simple harmonic motion}
Last class:
\begin{itemize}
\item Oscillations caused by restoring forces (e.g., stretch a spring, it pulls back)
\item Simple harmonic motion (i.e., sinusoidal oscillations) are caused by \textit{linear} restoring forces
\item Examples of linear restoring forces: masses on springs and pendulums (for small angles)
\item Derived equations of motion for simple harmonic motion
\end{itemize}

Displacement as a function of time:
$$x(t)=x_i\cos(2\pi ft)=x_i\cos\left(2\pi\frac{t}{T}\right)$$ 
Here $x_i$ is the initial amplitude of the oscillations and $f$ is the frequency.

Velocity is the instantaneous slope of position. We saw that this means that
$$v(t)=-v_{max}\sin(2\pi ft),$$
where $v_{max}=2\pi f x_i$.

Accleration is the slope of velocity. Graphically, we saw that this means that
$$a(t)=-a_{max}\cos(2\pi ft),$$
where $a_{max}=(2\pi f)^2x_i$.

\subsubsection{Springs}
\begin{itemize}
\item $f=\frac{1}{2\pi}\sqrt{\frac{k}{m}}$ 
\item From conservation of energy: $v_{max}=x_i\sqrt{\frac{k}{m}}$
\item From Newton's Second Law: $a_{max}=x_i\frac{k}{m}$

\end{itemize}

\subsubsection{Pendulums}
For pendulums, we use angular position instead of linear position and use the small angle approximation. Thus, we replace $x$, $v$, and $a$ with $\theta$, $\omega$, and $\alpha$. The equations work out to be identical (with different variables) to the equation for an oscillating spring. We found that
\begin{itemize}
\item for a simple pendulum, $f=\frac{1}{2\pi}\sqrt{\frac{g}{L}}$ 
\item for a physical pendulum, $f=\frac{1}{2\pi}\sqrt{\frac{mgl}{I}}$, where $I$ is the moment of inertia around the axis of rotation and $d$ is the distance from the axis of rotation to the center of mass of the pendulum
\end{itemize}

\subsubsection*{Example \#1: Pendulum on Mars}
The first astronauts to visit Mars are each allowed to take along some personal items to remind them of home. One astronaut takes along a grandfather clock, which, on Earth, has a pendulum that takes 1 second per swing, each swing corresponding to one tick of the clock. When the clock is set up on Mars, will it run faster or slower? How much faster or slower?
$$T=\frac{1}{f}=2\pi\sqrt{\frac{L}{g}}$$
Gravitational acceleration is lower on Mars (3.711 m/s$^2$), so the pendulum will run slower (the period of one oscillation is longer). To find how much slower, we need to determine the length of the pendulum.

$$L = g\left(\frac{T}{2\pi}\right)^2 = 0.25\mbox{ m}$$

$$T_{Mars} = 2\pi\sqrt{\frac{0.25\mbox{ m}}{3.711\mbox{ m/s}^2}} = 1.63\mbox{ s}$$

The clock runs 63\% slower. So when 24 hours have passed, the clock will only think that 14.7 hours have passed.

We could also determine this more directly by computing the ratio of the the period on Mars to the period on Earth:
$$\frac{T_{Mars}}{T_{Earth}} = \frac{2\pi\sqrt{\frac{L}{g_{Mars}}}}{2\pi\sqrt{\frac{L}{g_{Earth}}}} = \sqrt{\frac{g_{Earth}}{g_{Mars}}}=1.63$$
The advantage to this method is that it shows that the result is independent of the pendulum length and only depends on the ratio of the gravitational accelerations.

\subsubsection*{Example \#2: Oscillating spring}
 A 204 g block is suspended from a vertical spring, causing the spring to stretch by 20 cm. The block is then pulled down an additional 10 cm and released. What is the speed of the block when it is 5.0 cm above the equilibrium position?

 $$v=v_{max}\sin(2\pi ft)$$

 Note that I've dropped the negative sign from the above equation, which is because the initial displacement was negative.

 $$f=\frac{1}{2\pi}.\sqrt\frac{k}{m}$$
 $$v_{max}=x_i\sqrt{\frac{k}{m}}$$

 This problem involves several steps. We need to find $v_{max}$, $f$, and $t$ to solve for the speed. However, $f$ and $v_{max}$ both depend on $k$, so we also need to figure that out. Let's start with that.

 In equilibrium,
 $$\sum F_y = F_{sp}-F_g = -k\Delta y-mg = 0$$
 $$k=\frac{-mg}{\Delta y}$$
 Since $\Delta y=-0.1$~m and $m=0.204$~kg, we find that $k=20$~N/m. This means that $f=1.58$~Hz and $v_{max}=0.99$~m/s. 

 Now we need to figure out $t$ when $y=0.05$~m.
 $$y=y_i\cos(2\pi ft)$$
 $$t=\frac{1}{2\pi f}\cos^{-1}\left(\frac{y}{y_i}\right)=0.11\mbox{ s}$$

 Now plug this all back into the equation for $v$
 $$\boxed{v=0.86\mbox{ m/s}}$$
 
\subsection{Damped oscillations}
We've so far only discussed simple harmonic oscillators in the absence of external forces. External forces can cause oscillations to be damped or to grow. In damped oscillations, mechanical energy is converted to thermal energy. In driven oscillations, an external force does work on the system.

For a pendulum, the main energy loss is due to air resistance, which depends on speed. Let's include this drag force in the force-balance equation. Summing the forces in the direction of the pendulum's motion,
$$\sum F=-F_g\sin\theta-F_d=ma_t$$
The drag force is proportional to the pendulum's velocity
$$F_d=cv_t$$
where $c$ is a constant that depends on the density of the fluid (air) and the cross-sectional area and shape of the mass hanging from the pendulum. Inserting this into the above equation,
$$-mg\sin\theta-cv_t=ma_t$$
Note that $v_t$ and $a_t$ are the tangential velocity and acceleration. Recall that $v=\omega r = \omega L$ and $a_t = \alpha r = \alpha L$. Thus, our force balance equation becomes
$$-mg\sin\theta - cL\omega = mL\alpha$$
Using the small-angle approximation, this reduces to
$$-mg\theta - cL\omega = mL\alpha$$

We need differential equations to solve this.

Note that angular velocity is the rate of change of angular position, and angular acceleration is the rate of change of angular velocity. This gives
$$-mg\theta-cL\frac{d\theta}{dt}=mL\frac{d^2\theta}{dt^2}$$
which is a second-order ordinary differential equation. This one has the solution
$$\boxed{\theta(t)=\theta_ie^{-\frac{c}{2m}t}\cos\left(\sqrt{\frac{g}{L}-\frac{c^2}{4m^2}}t\right)}$$

I admit, this is a mess! But it tells us two things:

(1) The amplitude decays exponentially with time according because of the term $e^{-\frac{c}{2m}t}$. The $e$-folding time, which is the time at which the amplitude is $1/e$ of the original amplitude, is $t=2m/c$. For a metal sphere with a diameter of 0.01~m, $t\approx 400\mbox{ s}$. This is why pendulums can oscillate for a long time. You can increase the $e$-folding time by increasing the mass of the pendulum, so that air resistance has less of an impact on its motion.

(2) The natural frequency of the pendulum is 
$$f=\frac{1}{2\pi}\sqrt{\frac{g}{L}-\frac{c^2}{4m^2}}$$
Including drag decreases the frequency ever so slightly. For a 2-cm diameter steel ball, $c^2/(4m^2)\approx 2.7\times 10^{-9}\mbox{ s}^{-2}$. If $L=1\mbox{ m}$, then $g/L\approx 10\mbox{ s}^{-2}$. So the drag has only a very minor effect on the pendulum's frequency.

[Diagram of a damped oscillation.]\nopagebreak
\vspace{5cm}

\subsubsection*{Example \#3: Damped pendulum}
A 500 g mass on a string oscillates as a pendulum. The pendulum’s energy decays to 50\% of its initial value in 30 s. What is the value of the damping constant?

This question is essentially asking when does $e^{-\frac{c}{2m}t}=0.5$? In this case, we aren't told anything about the drag coefficient, so we have to think of this more generally. Instead, this could be written as $e^{-\frac{t}{\tau}}=0.5$, where $\tau$ is the damping constant or e-folding time. Solving for $\tau$:
$$\tau = -\frac{t}{\ln 0.5}=-\frac{30\mbox{ s}}{\ln 0.5} = 43.2\mbox{ s}$$

\subsection{Driven oscillations and resonance}
Oscillating systems have a natural frequency when left alone (as we've already seen). What happens if you subject a system to a periodic, driving frequency?

If the natural frequency matches the natural frequency of the system, you get large oscillations (especially if damping is small) --- this is called resonance. Modeling this is more difficult than modeling damped oscillations, which required differential equations, so let's just look at it conceptually.

[Frequency-response curve.]\nopagebreak
\vspace{5cm}


\subsubsection*{Demo: Pendulums of different lengths on a rod}


\subsubsection*{Example \#4: Cars on springs}
Cars ride on springs; they therefore have a natural frequency. For a particular car, the natural frequency is 2 Hz. The car is driving 20 mph over a washboard road with bumps every 10 ft. What is the driving frequency? How does it compare to the natural frequency?

$$v=\frac{\Delta{x}}{\Delta{t}}\Rightarrow T=\frac{\Delta{x}}{v}=\frac{10\mbox{ ft}}{20\mbox{ mph}}=0.35\mbox{ s}$$
Which means that
$$f=\frac{1}{T}=2.9\mbox{ Hz}$$

This is just above the natural frequency. Driving faster will reduce the amplitude of the oscillations.


\clearpage
