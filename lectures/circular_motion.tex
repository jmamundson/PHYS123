\section{CIRCULAR MOTION}
Objectives:
\begin{enumerate}
\item Definitions
\item Kinematic equations
\item Transforming between angular quantities and linear quantities
\end{enumerate}

We have covered straight line motion and projectile motion. This week we'll describe circular motion and rotational motion. Then we'll ask what causes motion -- or rather, changes in motion.


\subsection*{Basic definitions}
[Demo: ball on a string]

To describe motion around a circle, it is convenient to define position using the angle from the positive $x-$axis.

[Insert diagram defining $\theta$.]
\vspace{5cm}

Angular position is defined as $\theta=s/r$, in radians. Radians are \textit{not} units! However, it is sometimes convenient to report a measurement as being in radians so that it is clear that the number represents an angle. $\theta>0$ if counterclockwise from positive $x-$axis. Sometimes we will want to know the distance that an object has travelled along a circle, in which case we might use $s=r\theta$.

How do we convert between radians and degrees? $1\mbox{ rad}=\pi/180$

From here, we can probably guess how to define angular displacement, angular velocity, and angular acceleration.

Angular displacement: $\Delta{\theta}=\theta_f-\theta_i$\\
Angular velocity: $\omega=\frac{\Delta\theta}{\Delta{t}}$, [1/s]\\
Angular speed or angular frequency: $|\omega|$, [1/s]\\
Angular acceleration: $\alpha=\frac{\Delta\omega}{\Delta{t}}$, [1/s$^2$]

Like before, angular velocity is the slope of the angular position-versus-time graph, and angular acceleration is the slope of the angular velocity-versus-time graph. Conversely, angular displacement is the area under the angular velocity-versus-time graph, and the change in angular velocity is the area under the angular acceleration-versus-time graph. 

What is meant by positive or negative angular velocity? What is meant by positive or negative angular acceleration?

Sometimes we will need to convert between angular frequency and frequency.
$$\omega=2\pi/T=2\pi f$$
 
\subsection*{Kinematic equations for circular motion}
I won't go through the derivations for the kinematic equations for circular motion because they are exactly analogous to what we did for one-dimensional motion. For constant angular acceleration:
$$\Delta\omega=\alpha\Delta{t}$$
$$\Delta\theta=\omega_i\Delta{t}+\frac{1}{2}\alpha\Delta{t}^2$$
$$\omega_f\ds^2-\omega_i\ds^2=2\alpha\Delta\theta$$
When $\alpha=0$, this reduces to
$$\Delta\theta=\omega\Delta{t}.$$

\subsection*{Relating angular quantities to linear quantities}
Often we'll want to be able to convert between angular quantities and linear quantities. This derivation requires calculus, so I will just give you the results and we can focus on interpreting the equations.

For all of the following we will assume that the radius is constant.

\textbf{Position:}
$$x=r\cos\theta$$
$$y=r\sin\theta$$

Or, in vector notation:

$$\boxed{\vec x=\langle{r\cos\theta,r\sin\theta}\rangle}$$

\textbf{Velocity:}
$$\boxed{\vec{v}=\langle{-r\omega\sin\theta,r\omega\cos\theta}\rangle}$$

The speed is the magnitude of the velocity:
$$v=|\vec{v}|=\sqrt{(-r\omega\sin\theta)^2+(r\omega\cos\theta)^2}=\sqrt{r^2\omega^2(\sin^2\theta+\cos^2\theta)}$$
$$\boxed{|\vec{v}|=|\omega| r} \mbox{ or, in shorter notation } \boxed{v=\omega r}$$

\clearpage
[Insert diagram of velocity vector.]
\vspace{5cm}

\textbf{Acceleration:}
The acceleration consists of two parts: a component that cause the velocity vector to change direction (centripetal acceleration), and a component that cause the magnitude of the velocity vector to change (tangential acceleration).

Let's seperate the net acceleration into a centripetal acceleration and a tangential acceleration.

$$\vec a=\langle{-r\omega^2\cos\theta,-r\omega^2\sin\theta}\rangle+\langle{-r\alpha\sin\theta,r\alpha\cos\theta}\rangle=\vec{a}_c+\vec{a}_t$$

Note that the centripetal acceleration is non-zero as long as the object is moving in a circle. The tangential acceleration is only non-zero when the object's speed is changing.

[Insert diagram showing direction of the acceleration vectors, and pointing out that $\vec{a}_c$ always points inward and that $\vec{a}_c\perp\vec{a}_t$.]
\vspace{8cm}

The magnitude of the centripetal acceleration is
$$|\vec{a}_c|=\sqrt{(-r\omega^2\cos\theta)^2+(-r\omega^2\sin\theta)^2)}=\sqrt{r^2\omega^4(\cos^2\theta+\sin^2\theta)}$$
$$\boxed{a_c=\omega^2r=\frac{v^2}{r}}$$

The magnitude of the tangential acceleration is
$$|\vec{a}_t|=\sqrt{(-r\alpha\sin\theta)^2+(r\alpha\cos\theta)^2}=\sqrt{r^2\alpha^2(\sin^2\theta+\cos^2\theta)}$$
$$\boxed{a_t=\alpha r}$$

The centripetal acceleration and tangential acceleration are perpendicular to each other. This means that you can add them together to find the magnitude of the net acceleration by using the Pythagorean theorem.

$$a_{net}=\sqrt{a_c^2+a_t^2}$$
$$a_{net}=\sqrt{(\omega^2 r)^2+(\alpha r)^2}$$
$$\boxed{a_{net}=r\sqrt{\omega^4+\alpha^2}}$$

\subsection*{Example problem: disk in a hard drive}
Ex: The disk in a hard drive in desktop computer rotates at 7200 rpm. The disk has a radius of 13 cm. What is the angular speed of the disk?

Given:\\
$f=7200\mbox{ rpm}=120\mbox{ rev/s}$\\
$r=0.13\mbox{ m}$

$$\omega{r}=2\pi f=750\mbox{ s}^{-1}$$

\subsection*{Example problem: shaft of an elevator motor}
Ex: The shaft of an elevator motor turns clockwise at 180 rpm for 10 s, is at rest for 15 s, then turns counterclockwise at 240 rpm for 12.5 s. What is the angular displacement of the shaft during this motion. Draw angular position and angular velocity graphs for the shaft's motion.

Angular velocity during the first interval: $\omega_1=-3\mbox{ rad/s}$\\
Angular displacement during the first interval: $\Delta\theta_1=\omega_1\Delta{t_1}=-30\mbox{ rad}$

Angular velocity during the third interval: $\omega_3=4\mbox{ rad/s}$\\
Angular displacement during the third interval: $\Delta\theta_3=\omega_3\Delta{t_3}=50\mbox{ rad}$

Net displacement: 20 rad

[Insert position-time and velocity-time graphs.]


\clearpage
