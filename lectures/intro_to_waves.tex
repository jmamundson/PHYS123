\section{Introduction to waves}
Objectives:
\begin{itemize}
\item Wave model
  \begin{itemize}
    \item Oscillations of fixed points
    \item Snapshots in time
    \item Wave speed
  \end{itemize}
\end{itemize}

\subsection{Background}
Waves are a type of oscillation. There are many types of waves:
\begin{itemize}
\item sound waves -- gas
\item elastic waves -- solid
\item water waves -- liquid
\item electromagnetic waves -- don't require a medium...
\end{itemize}

These can be categorized as two different types of waves.

(1) Mechanical waves:
\begin{itemize}
\item Motion of a substance caused by a disturbance to the substance. Mechanical waves include sound, elastic, and water waves.
\item Wave speed, $v$, depends on material properties.
\item No net motion of material. Waves transmit energy, not particles. (Demonstrate with a slinky.)
\end{itemize}

(2) Electromagnetic waves:
\begin{itemize}
\item Waves of an electromagnetic field. Includes visible light, radio waves, x-rays, ...
\item Not due to the motion of a substance. EM waves can travel through vacuums!
\end{itemize}

The details of wave behavior differ, but we can discuss basic properties of waves with a ``wave model''.

The most basic waves are transverse (wave on a string) or longitudinal (sound wave). For transverse waves, the particle motion is perpendicular to the wave direction, whereas for longitudinal waves the particle motion is parallel to the wave direction.

Waves can travel as a single pulse or a series of pulses. When waves are produced by a simple harmonic oscillator, the waves are sinusoidal.

[Diagram of a sinusoidal wave; show how its position changes at some later time.]\nopagebreak
\vspace{5cm}


We can describe the wave displacement, $y$, by

$$y(x,t)=A\cos\left(2\pi\frac{x}{\lambda}\pm 2\pi\frac{t}{T}\right)$$

The displacement depends on both location and time. $x/\lambda$ describes the wave shape. The $+$ indicates that the wave is travelling to the left, the $-$ indicates that it is travelling to the right. Note that the wave model uses three dimensions ($x$, $y$, and $t$)... Also note that this is written in terms of a transverse wave, but it also applies to longitudinal waves. The vertical displacement can just be replaced with a horizontal displacement relative to an object's initial position.

Let's analyze this equation in a bit of detail. 

[Sketches of the waves.]\nopagebreak
\vspace{5cm}

\subsection{Oscillations of fixed points}
First, consider fixed points along the wave, e.g., $x=0$. Then
$$y(0,t)=A\cos\left(\pm 2\pi\frac{t}{T}\right)$$

Each particle in the waves oscillates sinusoidally with amplitude $A$.

What if $x=\lambda/2$?
$$y\left(\frac{\lambda}{2},t\right)=A\cos\left(2\pi\frac{\lambda/2}{\lambda}\pm 2\pi\frac{t}{T}\right)=A\cos\left(\pi\pm 2\pi\frac{t}{T}\right)$$
The $\pi$ at the beginning is just a ``phase'' shift. To see how, let's recall an identity from trigonometry:
$$\cos(\alpha\pm\beta)=\cos\alpha\cos\beta\mp\sin\alpha\sin\beta$$
Here, $\alpha=\pi$; $\cos\pi=-1$ and $\sin\pi=0$. This means that
$$y\left(\frac{\lambda}{2},t\right)=-A\cos\left(\pm 2\pi\frac{t}{T}\right)$$
which is what we would expect. If we looked at $x=\lambda$, we would find that 
$$y(\lambda,t)=A\cos\left(\pm 2\pi\frac{t}{T}\right)$$
which is the same as for $x=0$. This is good!

[Diagram showing how the points oscillate in time.]\nopagebreak
\vspace{5cm}

\subsection{Snapshots in time}
Now let's take a look at snapshots in time for a wave that is travelling to the right (use negative sign in the equation). If $t=0$, this means that
$$y(x,0)=A\cos\left(2\pi\frac{x}{\lambda}\right)$$
When $t=T/4$, this gives
$$y\left(x,\frac{T}{4}\right)=A\cos\left(2\pi\frac{x}{\lambda}-\frac{\pi}{2}\right),$$
which represents a 90$^\circ$ phase shift to the right. To see this, we can again use the angle sum and difference identity. In this case, $\beta=\pi/2$, so $\cos\beta=0$ and $\sin\beta=1$. This means that
$$y\left(x,\frac{T}{4}\right)=A\sin\left(2\pi\frac{x}{\lambda}\right)$$

[Diagram showing a wave at different points in time.]\nopagebreak
\vspace{5cm}

\subsection{Wave speed}
The wave model characterizes the speed at which a wave propagates to the left or right.
$$v=\frac{\Delta{x}}{\Delta{t}}=\frac{\lambda}{T}\Rightarrow \lambda=vT$$
Plugging this into the wave model gives
$$y(x,t)=A\cos\left(2\pi\frac{x}{vT}\pm 2\pi\frac{t}{T}\right)=A\cos\left(2\pi\frac{1}{T}\left(\frac{x}{v}\pm t\right)\right)=A\cos\left(2\pi f\left(\frac{x}{v}\pm t\right)\right)$$

This clearly shows that the angular frequency is the same if you are talking about a peak traveling a distance $\lambda$ in time $T$ or if you are talking about a fixed point going through one oscillation.

For some systems, the wave speed is independent of frequency. If this is the case, if you know the frequency or period you can calculate the wavelength, and vice-versa.

high frequency waves = short wavelength = high pitch (sound) or color blue (EM waves)

low frequency waves = long wavelength = low pitch (sound) or color red (EM waves)

\subsubsection{Wave on a string}
From a force balance analysis:
$$v=\sqrt{\frac{F_t}{\mu}},$$
where $F_t$ is the tensional force and $\mu$ is the linear density (mass / length). Waves travel quickly through taut strings that don't have much mass.

\subsubsection{Wave travelling through an ideal gas}
From thermodynamics and analysis of ideal gas law:
$$v_{sound}=\sqrt{\frac{\gamma RT}{M}}$$
where $\gamma=c_p/c_v$ is the adiabatic index and is often close to one, $R=8.31\mbox{ J/(mol}\cdot\mbox{K)}$ is the gas constant, $T$ is temperature in Kelvin, and $M$ is the molar mass (kg/mol of the gas). The speed of sound basically depends on temperature.

For non-ideal gases, $v_{sound}$ has a slight dependency on density, and therefore pressure, but we won't worry about that.

\subsubsection{Electromagnetic waves}
Predicted by Maxwell, and has not been proven otherwise. 
In a vacuum,
$$v=c\approx 3\times 10^8\mbox{ m/s}=\mbox{constant}$$

In other materials, 
$$v=\frac{c}{n},$$
where $n\geq 1$ is the index of refraction. It essentially has to do with interactions between the electromagnetic wave and electrons in the material.



\clearpage
