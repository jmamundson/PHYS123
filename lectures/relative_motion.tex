\section{Relative motion}
Objectives:
\begin{enumerate}
\item Relative position, velocity, and acceleration in 1-D and 2-D
\end{enumerate}

\subsection{Relative motion in 1-D}
In physics we often encounter problems where we need to know the motion of an object relative to another object. In order to talk about relative motion, we need to be comfortable with vectors.

Visit ophysics.com to do exercises on vector components and vector addition and subtraction.


\begin{enumerate}
\item Motion of an object is always relative to some reference frame. 
\item Reference frames \textit{can} move. (If a reference frame is accelerating, then we have to use Einstein's theory of general relativity.)
\end{enumerate}

You've all experienced relative motion in one way or another. For example, if you're sitting in a parked car, and the car next to you starts to move, you sometimes get the sensation that you've started moving.

We are going to start with relative motion in one-dimension, because this is easier to understand than motion in two-dimensions.

For example, Alex and Barbara each have their own reference frame; they are located at the origin of their reference frames. We will assume that Alex and Barbara are stationary and facing the same direction. There is an object located at point P; it also has a reference frame that is oriented in the same direction as Alex and Barbara's reference frames.

[Insert diagram of Alex and Barbara and object at point P.]
\vspace{4cm}


For Alex, $P$ is at a positive position, whereas $P$ is at a negative position for Barbara. We can relate the relative position of $P$ to Alex with the relative position of $P$ to Barbara.
$$x_{pa}=x_{pb}+x_{ba}$$
The order of indices matters. Note that $x_{pa}=-x_{ap}$. Why does this expression make sense?

What if point $P$ is moving? Can we write down a similar expression using velocities? What are $v_{pa}$ and $v_{pb}$?
$$v_{pa}=\frac{\Delta{x_{pa}}}{\Delta{t}}$$
$$v_{pb}=\frac{\Delta{x_{pb}}}{\Delta{t}}$$
$$v_{ba}=\frac{\Delta{x_{ba}}}{\Delta{t}}$$
We already saw that $x_{pa}=x_{pb}+x_{ba}$, so
$$v_{pa}=\frac{\Delta(x_{pb}+x_{ba})}{\Delta{t}}=\frac{\Delta x_{pb} + \Delta x_{ba}}{\Delta t} = \frac{\Delta{x_{pb}}}{\Delta{t}}+\frac{\Delta{x_{ba}}}{\Delta{t}}$$
$$\boxed{v_{pa}=v_{pb}+v_{ba}}$$

What if point $P$ is accelerating? What is the relative acceleration of point $P$?
$$a_{pa}=\frac{\Delta{v_{pa}}}{\Delta{t}}$$
$$a_{pb}=\frac{\Delta{v_{pb}}}{\Delta{t}}$$
$$a_{ba}=\frac{\Delta{x_{ba}}}{\Delta{t}}$$
And now combining these, like we did when going from displacement to velocity,
$$\frac{\Delta{v_{pa}}}{\Delta{t}}=\frac{\Delta(v_{pb}+v_{ba})}{\Delta t} = \frac{\Delta{v_{pb}}+\Delta{v_{ba}}}{\Delta{t}}=\frac{\Delta{v_{pb}}}{\Delta{t}}+\frac{\Delta{v_{ba}}}{\Delta{t}}$$
$$a_{pa}=a_{pb}+a_{ba}$$

We will only deal with situations in which the reference frames are NOT accelerating. This means that $a_{ba}=0$, and therefore that 
$$\boxed{a_{pb}=a_{pa}}$$

Furthermore, if $A$ and $B$ are moving at the same velocity, i.e. $v_{ba}=0$, then
$$\boxed{v_{pa}=v_{pb}}$$

\subsection{Example \#1: Boat traveling upriver}
A boat travels upriver at 14 km/h relative to the water. The water flows 9 km/h relative to the ground.

(a) What are the magnitude and direction of the boat's velocity relative to the ground?\\
(b) A child walks to from the bow to the stern at 6 km/h. What are the magnitude and direction of the child's velocity relative to the ground?

Have students work on these.

(a) Given: $v_{bw}=-14\mbox{ km/h}$; $v_{wg}=9\mbox{ km/h}$. Want to find $v_{bg}$.
$$v_{bg}=v_{bw}+v_{wg}=-14+9=-5\mbox{ km/h}$$
So the boat travels at $5$ km/h upstream.

(b) Given: $v_{bg}=-5\mbox{ km/h}$; $v_{cb}=6\mbox{ km/h}$. Want to find $v_{cg}$.
$$v_{cg}=v_{cb}+v_{bg}=6-5=1\mbox{ km/h}$$

\subsection{Relative motion in 2-D}
Relative motion is exactly identical in two dimensions, but now we have to use vectors. So, for example,
$$\vec{x}_{pa}=\vec{x}_{pb}+\vec{x}_{ba}.$$

You are trying to cross a river with a boat and would like to be exactly on the opposite side of the river. Your boat can travel 20 m/s (relative to the water). The river is flowing 2 m/s and is 1000 m wide. What angle should you leave shore at, and how long will it take you to reach the other side? (Note that ophysics has a simulation that shows this as well.)

[Insert diagram showing what happens if you go straight across the river.]
\vspace{4cm}

If you head straight across the river, it will take you 50 s to cross the river ($\Delta{t}=\Delta{x}/v_x$), you will end up 100 m downstream from your objective.

[Insert diagram showing angle $\theta$.]
\vspace{4cm}

$$\vec{v}_{bw}+\vec{v}_{wo}=\vec{v}_{bo}$$
$$\langle{V\cos\theta,V\sin\theta}\rangle+\langle{0,-V_w}\rangle=\langle{V_o,0}\rangle$$

We want to find $\theta$ and $V_o$; the latter will tell us how long it takes to cross the river. This vector equation is the same thing as writing down two equations (with two unknowns).
$$V\cos\theta=V_o$$
$$V\sin\theta-V_w=0$$
From the second equation,
$$\theta=\sin\ds^{-1}\left(\frac{V_w}{V}\right)\approx 5.7^\circ$$
Inserting this into the first equation gives
$$V_o=19.9\mbox{ m/s}$$
So,
$$\Delta{t}=\frac{\Delta{x}}{V_o}=\frac{1000\mbox{ m}}{19.9 \mbox{ m/s}}=50.3\mbox{ s}$$

\subsection{Example \#2: hockey player and puck}
A hockey player is skating due south at 7.0 m/s. A puck is passed to him with a speed of 11.0 m/s and direction 22$^\circ$ west of south. What are the magnitude and direction (relative to due south) of the puck's velocity, relative to the hockey player?

Given:\\
$V_h=7.0\mbox{ m/s}$\\
$V_p=11.0\mbox{ m/s}$ at angle of $22^\circ$ west of south

We want to know $\vec{v}_{ph}$.

$$\vec{v}_{ph}=\vec{v}_{pi}+\vec{v}_{ih}=\vec{v}_{pi}-\vec{v}_{hi}$$
$$\vec{v}_{ph}=\langle{-V_p\sin\theta,-V_p\cos\theta}\rangle+\langle{0,-V_h}\rangle=\langle{-V_p\sin\theta,-V_p\cos\theta}\rangle-\langle{0,-V_h}\rangle$$
$$\vec{v}_{ph}=\langle{-V_p\sin\theta,-V_p\cos\theta+V_h}\rangle=\langle{-4.12,-3.2}\rangle$$

speed = 5.2 m/s\\
angle = 52.2$^\circ$ 

\clearpage
