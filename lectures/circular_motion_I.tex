\section{Circular and rotational motion I}
Objectives:
\begin{enumerate}
\item Definitions
\item Kinematic equations
\item Rolling motion
\end{enumerate}

We have covered straight line motion and projectile motion. This week we'll describe circular motion and rotational motion. Then we'll ask what causes motion---or rather, changes in motion.

\subsection{Basic definitions}
[Demo: ball on a string]

To describe motion around a circle, it is convenient to define position using the angle from the positive $x-$axis.

[Insert diagram defining $\theta$.]
\vspace{5cm}

Angular position is defined as $\theta=s/r$, in radians. Radians are \textit{not} units! However, it is often convenient to report a measurement as being in radians so that it is clear that the number represents an angle. $\theta>0$ if counterclockwise from the positive $x-$axis. Sometimes we will want to know the distance that an object has travelled along a circle, in which case we might use $s=r\theta$. (Next class we'll relate angular quantities to linear quantities.)

How do we convert between radians and degrees? $1\mbox{ rad}=\pi/180$

From here, we can probably guess how to define angular displacement, angular velocity, and angular acceleration.

Angular displacement: $\Delta{\theta}=\theta_f-\theta_i$, [rad]\\
Angular velocity: $\omega=\frac{\Delta\theta}{\Delta{t}}$, [rad/s]\\
Angular speed or angular frequency: $|\omega|$, [rad/s]\\
Angular acceleration: $\alpha=\frac{\Delta\omega}{\Delta{t}}$, [rad/s$^2$]

Like before, angular velocity is the slope of the angular position-versus-time graph, and angular acceleration is the slope of the angular velocity-versus-time graph. Conversely, angular displacement is the area under the angular velocity-versus-time graph, and the change in angular velocity is the area under the angular acceleration-versus-time graph. 

What is meant by positive or negative angular velocity? What is meant by positive or negative angular acceleration?
 
\subsection{Kinematic equations for circular motion}
I won't go through the derivations for the kinematic equations for circular motion because they are exactly analogous to what we did for one-dimensional motion. For constant angular acceleration:
$$\Delta\omega=\alpha\Delta{t}$$
$$\Delta\theta=\omega_i\Delta{t}+\frac{1}{2}\alpha\Delta{t}^2$$
$$\omega_f\ds^2-\omega_i\ds^2=2\alpha\Delta\theta$$
When $\alpha=0$, this reduces to
$$\Delta\theta=\omega\Delta{t}.$$

Note that if $\Delta t=T$, where $T$ is the period (i.e., the time to complete one revolution), then $\Delta \theta = 2\pi$ and therefore
$$2\pi = \omega T \Rightarrow \omega = \frac{2\pi}{T}$$
Frequency is defined as the number of revolutions per unit time and is given by $f=1/T$, which means that we can also write that
$$\omega = 2\pi f$$

\subsection{Example \#1: shaft of an elevator motor}
The shaft of an elevator motor turns clockwise at 180 rpm for 10 s, is at rest for 15 s, then turns counterclockwise at 240 rpm for 12.5 s. What is the angular displacement of the shaft during this motion. Draw angular position and angular velocity graphs for the shaft's motion.

Angular velocity during the first interval: $\omega_1=-6\pi\mbox{ rad/s}$\\
Angular displacement during the first interval: $\Delta\theta_1=\omega_1\Delta{t_1}=-60\pi\mbox{ rad}$

Angular velocity during the third interval: $\omega_3=8\pi\mbox{ rad/s}$\\
Angular displacement during the third interval: $\Delta\theta_3=\omega_3\Delta{t_3}=100\pi\mbox{ rad}$

Net displacement: 40$\pi$ rad

[Insert position-time and velocity-time graphs.]

\clearpage
\subsection{Rolling motion (w/out slipping)}
Rolling of wheels and other objects involves rotational motion. Depending on our frame of reference, we observe it either as rotation around the point where the wheel touches the ground or the center of the wheel. Let's first consider this from a static reference frame to see how the wheel's velocity is related to the angular velocity.

[Derive these equations using a demo.]

When the wheel has undergone one revolution, the center of the wheel will have traveled a distance equal to the circumference of the wheel, meaning that
$$\Delta x = 2\pi R$$
For the case of uniform motion (i.e., constant velocity),
$$v = \frac{\Delta x}{\Delta t}$$
If the wheel has undergone one revolution, then $\Delta t = T$. Putting this together,
$$v = \frac{2\pi}{T}R = \omega R,$$
which is referred to as the rolling constraint. Note that $\omega$ is the angular velocity of the wheel around it's axis, but in our reference frame the wheel appears to be rotating around the point where it touches the ground.

[Diagram showing rolling motion in our reference frame.]
\vspace{3cm}

In our reference frame, rolling motion can be viewed as a sum of rotational and translation motion.

[Diagram depicting rotational + translational motion.]
\vspace{3cm}

Note that the velocity of the center of the wheel in our reference frame is the same as the tangential velocity for the rotational component of motion. If instead we are in a reference frame that moves with the wheel, we will only see rotational motion and the wheel will appear to rotate around its axis. 



\subsection{Example \#2: chain speed on a bicycle}
The wheels on a road bike are around 0.70 m in diameter. In order to travel 9~m/s (20 mph), how fast must the chain move? Consider two different sprockets, one with a diameter of 7~cm and another with a diameter of 14~cm. It is easiest to view this problem from the moving reference frame.

[Diagram of wheel and sprocket.]
\vspace{3cm}


The entire wheel must rotate with the same angular velocity. During one rotation, a point on the edge of the wheel travels a distance of $2\pi R$, whereas a point on the sprocket travels $2\pi r$. The velocity of the edge of the wheel is
$$v_w = \frac{2\pi R}{T} = \omega R$$
as we've already seen. The velocity of the point on the sprocket is
$$v_a = \frac{2\pi r}{T} = \omega r$$

Solving both for $\omega$ and setting them equal to each other,
$$\omega = \frac{v_a}{r} = \frac{v_w}{R}$$
$$v_a = v_w\frac{r}{R}$$

Plugging in values, we get 0.9~m/s for the small sprocket and 1.8~m/s for the large sprocket. The large sprocket is better for accelerating, but it is difficult to move fast with it because the chain must travel much faster than when it is on the small sprocket.



\clearpage
