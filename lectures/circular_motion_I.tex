\section{Circular and rotation motion I}
Objectives:
\begin{enumerate}
\item Definitions
\item Kinematic equations
\end{enumerate}

We have covered straight line motion and projectile motion. This week we'll describe circular motion and rotational motion. Then we'll ask what causes motion -- or rather, changes in motion.

\subsection{Basic definitions}
[Demo: ball on a string]

To describe motion around a circle, it is convenient to define position using the angle from the positive $x-$axis.

[Insert diagram defining $\theta$.]
\vspace{5cm}

Angular position is defined as $\theta=s/r$, in radians. Radians are \textit{not} units! However, it is sometimes convenient to report a measurement as being in radians so that it is clear that the number represents an angle. $\theta>0$ if counterclockwise from positive $x-$axis. Sometimes we will want to know the distance that an object has travelled along a circle, in which case we might use $s=r\theta$.

How do we convert between radians and degrees? $1\mbox{ rad}=\pi/180$

From here, we can probably guess how to define angular displacement, angular velocity, and angular acceleration.

Angular displacement: $\Delta{\theta}=\theta_f-\theta_i$\\
Angular velocity: $\omega=\frac{\Delta\theta}{\Delta{t}}$, [1/s]\\
Angular speed or angular frequency: $|\omega|$, [1/s]\\
Angular acceleration: $\alpha=\frac{\Delta\omega}{\Delta{t}}$, [1/s$^2$]

Like before, angular velocity is the slope of the angular position-versus-time graph, and angular acceleration is the slope of the angular velocity-versus-time graph. Conversely, angular displacement is the area under the angular velocity-versus-time graph, and the change in angular velocity is the area under the angular acceleration-versus-time graph. 

What is meant by positive or negative angular velocity? What is meant by positive or negative angular acceleration?

Sometimes we will need to convert between angular frequency and frequency.
$$\omega=2\pi/T=2\pi f$$
 
\subsection{Kinematic equations for circular motion}
I won't go through the derivations for the kinematic equations for circular motion because they are exactly analogous to what we did for one-dimensional motion. For constant angular acceleration:
$$\Delta\omega=\alpha\Delta{t}$$
$$\Delta\theta=\omega_i\Delta{t}+\frac{1}{2}\alpha\Delta{t}^2$$
$$\omega_f\ds^2-\omega_i\ds^2=2\alpha\Delta\theta$$
When $\alpha=0$, this reduces to
$$\Delta\theta=\omega\Delta{t}.$$



\subsection{Example \#1: shaft of an elevator motor}
The shaft of an elevator motor turns clockwise at 180 rpm for 10 s, is at rest for 15 s, then turns counterclockwise at 240 rpm for 12.5 s. What is the angular displacement of the shaft during this motion. Draw angular position and angular velocity graphs for the shaft's motion.

Angular velocity during the first interval: $\omega_1=-3\mbox{ rad/s}$\\
Angular displacement during the first interval: $\Delta\theta_1=\omega_1\Delta{t_1}=-30\mbox{ rad}$

Angular velocity during the third interval: $\omega_3=4\mbox{ rad/s}$\\
Angular displacement during the third interval: $\Delta\theta_3=\omega_3\Delta{t_3}=50\mbox{ rad}$

Net displacement: 20 rad

[Insert position-time and velocity-time graphs.]


\subsection{Example \#2: ???}


\clearpage
