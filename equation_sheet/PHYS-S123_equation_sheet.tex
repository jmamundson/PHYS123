\documentclass[11pt,letterpaper]{article}
\usepackage[top=1.5cm,bottom=2cm,left=1.5cm,right=1.5cm]{geometry}
\pagestyle{empty}
\renewcommand{\labelitemi}{$\cdot$}

\begin{document}
\noindent Basic definitions for linear motion
\begin{itemize}
\item displacement: $\Delta{\vec{x}}=\vec{x}_f-\vec{x}_i$
\item average velocity: $\vec{v}=\displaystyle\frac{\Delta{\vec{x}}}{\Delta{t}}$
\item average acceleration: $\vec{a}=\displaystyle\frac{\Delta{\vec{v}}}{\Delta{t}}$
\item if $a\cdot v > 0$, object is speeding up
\end{itemize}
\noindent Kinematic equations, $\vec{v}=\mbox{constant}$ (i.e., $\vec{a}=0$)
\begin{itemize}
\item $\vec{v}=\displaystyle\frac{\Delta{\vec{x}}}{\Delta{t}}$
\item $\vec{x}_f=\vec{x}_i+\vec{v}\Delta{t}$
\end{itemize}
\noindent Kinematic equations, $\vec{a}\neq{0}$ and $\vec{a}=\mbox{constant}$
\begin{itemize}
\item $\vec{a}=\displaystyle\frac{\Delta{\vec{v}}}{\Delta{t}}$
\item $\vec{v}_f=\vec{v}_i+\vec{a}\Delta{t}$
\item $\Delta{\vec{x}}=\vec{v}_i\Delta{t}+\displaystyle\frac{1}{2}\vec{a}\Delta{t}^2$
\item ${v_f}^2-{v_i}^2=2a\Delta{x}$
\item common example of $a=\mbox{constant}$ is $g=9.81\mbox{ m/s}^2$
\end{itemize}
\noindent Motion of object A relative to object C
\begin{itemize}
\item $\vec{v}_{ac}=\vec{v}_{ab}+\vec{v}_{bc}$
\end{itemize}

\noindent Basic definitions for circular and rotational motion
\begin{itemize}
\item angle: $\theta\mbox{(radians)}=\displaystyle\frac{s}{r}$; $s=$arclength, $r=$radius
\item angular displacement: $\Delta\theta=\theta_f-\theta_i$
\item angular velocity: $\omega=\displaystyle\frac{\Delta\theta}{\Delta{t}}=2\pi{f}=2\pi\displaystyle\frac{1}{T}$
\item angular acceleration: $\alpha=\displaystyle\frac{\Delta\omega}{\Delta{t}}$
\item if $\alpha \cdot \omega > 0$, object is speeding up
\end{itemize}
\noindent Kinematic equations
\begin{itemize}
\item $\Delta\theta=\omega_i\Delta{t}+\displaystyle\frac{1}{2}\alpha\Delta{t}^2$; if $\alpha=0$, $\omega_f=\omega_i=\mbox{constant}$.
\item $\omega_f^2-\omega_i^2=2\alpha\Delta{\theta}$
\end{itemize}
\noindent Speed, acceleration, and forces
\begin{itemize}
\item speed: $v=\omega{r}$
\item centripetal acceleration: $a_c=\displaystyle\frac{v^2}{r}=\omega^2r$
\item centripetal force: $F_c=ma_c=m\displaystyle\frac{v^2}{r}=m\omega^2r$; points toward center of circle
\item tangential acceleration: $a_t=\alpha{r}$
\end{itemize}

\noindent Newton's Laws
\begin{enumerate}
\item if $\vec{F}_{net}=0\Rightarrow \vec{a}=0$
\item $\displaystyle\sum\vec{F}=m\vec{a}$
\item $\vec{F}_{1\,on\,2}=-\vec{F}_{2\,on\,1}$
\end{enumerate}
\noindent Types of forces
\begin{itemize}
\item gravitational force (aka, weight) at the Earth's surface, $\vec{F}_g=m\vec{g}$
\item Newton's Law of Gravity (gravitational force between two objects): $F_{1\,on\,2}=F_{2\,on\,1}=\displaystyle\frac{Gm_1m_2}{r^2}$
\item Gravitational constant: $G=6.67\times{10}^{-11}\mbox{ N}\cdot\mbox{m}^2/\mbox{kg}^2$
\item normal force, $\vec{F}_n$, is perpendicular to surface and prevents objects from penetrating the surface
\item frictional force, $\vec{F}_f$, depends on $\vec{F}_n$ and $\vec{v}$
\begin{itemize}
\item if $\vec{v}=0$, $\vec{F}_f$ balances other forces as long as $\left|\vec{F}_f\right|\leq\mu_s\left|\vec{F}_n\right|$
\item if $\vec{v}\neq{0}$, $\left|\vec{F}_f\right|=\mu_k\left|\vec{F}_n\right|$
\end{itemize}
\item tensional force, $\vec{F}_t$, is transmitted through a rope and around pulleys
\item spring force / Hooke's `Law': $F=-k\Delta{x}$; $k$ is empirically determined spring constant
\end{itemize}


\noindent Torque and moment of inertia
\begin{itemize}
\item Torque: $\tau=rF_\perp$; $F_\perp$ is force perpendicular to radial axis
\item Newton's Second Law for rotation: $\sum\tau=I\alpha$
\item Moment of inertia: $I$, indicates how difficult it is to rotate an object
\item Rolling constraint: $v=\omega{R}$; rolling object, with perfect friction, moves forward with this velocity
\end{itemize}

\noindent Static equilibrium and elasticity
\begin{itemize}
\item $\sum{F_x}=0$; $\sum{F_y}=0$; $\sum\tau=0$
\item Choose convenient pivot point
\item Before coming to equilibrium, objects change shape to accommodate the forces being exerted on them.
\item The relationship between force and deformation for elastic materials is given by $\displaystyle\frac{F}{A}=Y\left(\displaystyle\frac{\Delta{L}}{L}\right)$, where $Y$ is Young's modulus (a material property). Sometimes this is written as $\mbox{stress}=Y\times\mbox{strain}$
\end{itemize}

\noindent Impulse and momentum
\begin{itemize}
\item impulse, $\vec{J}=\vec{F}_{avg}\Delta{t}$
\item momentum, $\vec{p}=m\vec{v}$
\item Impulse-Momentum Theorem, $\vec{J}=\Delta{\vec{p}}$
\item total momentum, $\vec{P}=\vec{p}_1+\vec{p}_2+...$
\item conservation of momentum, $\Delta{P}=0$ if $\vec{F}_{net}=0$ or if $\vec{F}_{net}$ is small compared to other forces for short $\Delta{t}$
\end{itemize}

\noindent Energy and work
\begin{itemize}
\item Total energy: $E=K+U_g+U_s+E_{ch}+E_{th}+...$
\item Work: $W=\Delta{E}$; mechanical transfer of energy into or out of a system
\item Conservation of energy: for isolated system, $W=0$ and therefore $\Delta{E}=0$.
\item Work: $W=F_\parallel\cdot{d}$; force $\times$ displacement
\item Translational kinetic energy: $K_{trans}=\displaystyle\frac{1}{2}mv^2$; scalar quantity
\item Rotational kinetic energy: $K_{rot}=\displaystyle\frac{1}{2}I\omega^2$
\item Total kinetic energy: $K=K_{trans}+K_{rot}$
\item Gravitational potential energy: $\Delta{U_g}=mg\Delta{y}$; choose convenient reference height
\item Elastic potential energy: $U_s=\displaystyle\frac{1}{2}k(\Delta{x})^2$; $k$ is the spring constant
\item Thermal energy (from friction): $\Delta{E_{th}}=F_f\Delta{x}$
\item Conservation of mechanical energy: for isolated system with no friction, $\Delta{K}+\Delta{U_g}+\Delta{U_s}=0$
%\item Power: $P=\displaystyle\frac{\Delta{E}}{\Delta{t}}=\displaystyle\frac{W}{\Delta{t}}=Fv$
\end{itemize}

Thermodynamics
\begin{itemize}
\item 1st Law: $Q+W=\Delta{E_{th}}$; $Q$ is heat transferred into or out of the system
\item 2nd Law: Entropy in a closed system (which describes disorder) can never decrease
\item energy needed to change material's temperature: $Q=mc\Delta{T}$; $c$ is the specific heat, a material property
\item energy need to change a material's phase: $Q=mL_f$ to melt solid and $Q=mL_v$ to turn liquid into gas
\item $L_v>L_f$ (latent heat of vaporization is greater than latent heat of fusion)
\item conduction: $\frac{Q}{\Delta{t}}=\left(\frac{kA}{L}\right)\Delta{T}$; $k$ is thermal conductivity
\item advection/convection: need to solve fluid flow equations
\item radiation: $\frac{Q}{\Delta{t}}=e\sigma{A}T^4$; $e$ is emissivity and $\sigma=5.67\times{10}^{-8}\frac{\mbox{W}}{\mbox{m}^2\mbox{K}^4}$ is the Stefan-Boltzmann constant
\item for ideal gas in enclosed, insulated container, $\frac{PV}{T}=\mbox{constant}$
\item work done by expanding a gas at constant pressure: $W_{\mathrm{gas}}=P\Delta{V}$
\item note that for gases, specific heat at constant pressure $c_p$ differs from specific heat at constant volume $c_v$. $c_p$ is approximately 50$\%$ larger than $c_v$
\item thermal energy: $\displaystyle\Delta E_{th} = \frac{3}{2}nR\Delta T$
\end{itemize}

\clearpage
Fluids
\begin{itemize}
\item density: $\rho=\frac{m}{V}$
\item hydrostatic pressure: $P=\rho{g}h+P_0$; $P_0$ is pressure from above
\item Archimede's principle (upward buoyant force): $F_b=\rho_f{g}V$; $\rho_f$ is fluid density and $V$ is volume displaced
\item flux (continuity): $Q=vA$; $Q$ is constant in a pipe for ideal fluid
\item Bernoulli's equation: along a streamline, $P+\frac{1}{2}\rho{v^2}+\rho{g}h=\mathrm{constant}$
\end{itemize}

Simple harmonic motion
\begin{itemize}
\item displacement: $x(t)=A\cos(2\pi{f}t)$
\item velocity: $v(t)=-v_{\mathrm{max}}\sin(2\pi{f}t)$
\item acceleration: $a(t)=-a_{\mathrm{max}}\cos(2\pi{f}t)$
\item $v_{\mathrm{max}}=2\pi{f}A$ and $a_{\mathrm{max}}=(2\pi{f})^2A$
\item for a spring, $f=\frac{1}{2\pi}\sqrt{\frac{k}{m}}$; for a pendulum, $f=\frac{1}{2\pi}\sqrt{\frac{g}{L}}$
\end{itemize}

Waves
\begin{itemize}
\item in general, $v=\lambda/T$, where $\lambda$ is wavelength and $T$ is period
\item velocity of a wave on a string: $v=\sqrt{\frac{F_t}{\mu}}$, where $\mu$ is linear density
\item velocity of sound waves in ideal gas: $v=\sqrt{\frac{\gamma{RT}}{M}}$, where $\gamma$ is adiabatic index, $R=8.314\frac{\mbox{J}}{\mbox{mol K}}$ is the gas constant, $T$ is temperature in Kelvin, and $M$ is the molar mass
\item travelling wave displacement: $y(x,t)=A\sin\left(2\pi\left(\frac{x}{\lambda}-\frac{t}{T}\right)\right)$
\item Doppler effect: $f=\frac{f_s}{1\pm{v_s}/v}$; $v_s$ is speed of source, $v$ is wave speed. Use $+$ if the source is moving away from the receiver, $-$ if the source is moving toward the receiver.
\item wave superposition: $y(x,t)=y_1+y_2+y_3$ where $y_i$ are given by different amplitudes, wavelengths, and frequencies/periods.
\item standing wave displacement: $y(x,t)=A\cos\left(2\pi\left(\frac{x}{\lambda}-\frac{t}{T}\right)\right)+A\cos\left(2\pi\left(\frac{x}{\lambda}+\frac{t}{T}\right)\right)$
\item harmonics: $f_m=\frac{2L}{m}$ for a string that is pinned on both ends
\end{itemize}




\end{document}
